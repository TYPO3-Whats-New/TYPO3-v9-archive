% ------------------------------------------------------------------------------
% TYPO3 Version 9.2 - What's New - Chapter "Changes for Developers" (English Version)
%
% @author	Michael Schams <schams.net>
% @license	Creative Commons BY-NC-SA 3.0
% @link		http://typo3.org/download/release-notes/whats-new/
% @language	English
% ------------------------------------------------------------------------------
% LTXE-CHAPTER-UID:		846d40ce-66fdc2ec-750dcf95-33ce93e0
% LTXE-CHAPTER-NAME:	Changes for Developers
% ------------------------------------------------------------------------------

\section{Änderungen für Entwickler}
\begin{frame}[fragile]
	\frametitle{Änderungen für Entwickler}

	\begin{center}\huge{Kapitel 3:}\end{center}
	\begin{center}\huge{\color{typo3darkgrey}\textbf{Änderungen für Entwickler}}\end{center}

\end{frame}

% ------------------------------------------------------------------------------
% LTXE-SLIDE-START
% LTXE-SLIDE-UID:		c7b4f7f2-a11e6fc7-4ffe6e04-92cc98aa
% LTXE-SLIDE-TITLE:		PSR-15 Middlewares Support (1)
% LTXE-SLIDE-REFERENCE:	Feature-83725-SupportForPSR-15HTTPMiddlewares.html
% ------------------------------------------------------------------------------

\begin{frame}[fragile]
	\frametitle{Änderungen für Entwickler}
	\framesubtitle{PSR-15 Middlewares Unterstützung (1)}

	\begin{itemize}
		\item TYPO3 möchte den \href{https://www.php-fig.org/psr/psr-15/}{PSR-15 standard} out-of-the box
			unterstützen
		\item Dies wird die Interoperabilität mit unabhängigen Bibliotheken verbessern und alle 
			Anfragen im TYPO3-Kern werden eine PSR-7-Reaktion zurückgeben

		\item Die PSR-15 Standard wird folgenderweise definiert:
			\newline
			\smaller
				\textit{[PSR-15] describes common interfaces for HTTP server request handlers
				(request handlers) and HTTP server middleware components (middleware) that use
				HTTP messages [...]. HTTP request handlers are a fundamental part of any web
				application. Server side code receives a request message, processes it, and
				produces a response message. HTTP middleware is a way to move common request
				and response processing away from the application layer."}
				\newline
				Siehe \url{https://www.php-fig.org/psr/psr-15/} für weitere Informationen.
			\normalsize

	\end{itemize}

	% Note for translators: do not translate the PSR-15 definition above.
	% It is a quote from the php-fig.org website and should remain as it stands.

\end{frame}

% ------------------------------------------------------------------------------
% LTXE-SLIDE-START
% LTXE-SLIDE-UID:		c7b4f7f2-a11e6fc7-4ffe6e04-92cc98aa
% LTXE-SLIDE-TITLE:		PSR-15 Middlewares Support (2)
% LTXE-SLIDE-REFERENCE:	Feature-83725-SupportForPSR-15HTTPMiddlewares.html
% ------------------------------------------------------------------------------

\begin{frame}[fragile]
	\frametitle{Änderungen für Entwickler}
	\framesubtitle{PSR-15 Middlewares Unterstützung (2)}

	% decrease font size for code listing
	\lstset{basicstyle=\tiny\ttfamily}

	\begin{itemize}
		\item Um eine Middleware zu der "\textit{frontend}" oder "\textit{backend}"
			Middleware-Stack hinzuzufügen, muss eine \texttt{Configuration/RequestMiddlewares.php}
			Datei in der jeweiligen Extension erstellt werden:

			\begin{lstlisting}
				return [
				  // stack name: currently 'frontend' or 'backend'
				  'frontend' => [
				    'middleware-identifier' => [
				      'target' => \ACME\Ext\Middleware::class,
				      'description' => '',
				      'before' => [
				        'another-middleware-identifier',
				      ],
				      'after' => [
				        'yet-another-middleware-identifier',
				      ],
				    ]
				  ]
				];
			\end{lstlisting}

	\end{itemize}

\end{frame}

% ------------------------------------------------------------------------------
% LTXE-SLIDE-START
% LTXE-SLIDE-UID:		c7b4f7f2-a11e6fc7-4ffe6e04-92cc98aa
% LTXE-SLIDE-TITLE:		PSR-15 Middlewares Support (3)
% LTXE-SLIDE-REFERENCE:	Feature-83725-SupportForPSR-15HTTPMiddlewares.html
% ------------------------------------------------------------------------------

\begin{frame}[fragile]
	\frametitle{Änderungen für Entwickler}
	\framesubtitle{PSR-15 Middlewares Unterstützung (3)}

	% decrease font size for code listing
	\lstset{basicstyle=\tiny\ttfamily}

	\begin{itemize}

		\item Wenn Erweiterungen stillgelegt werden müssen oder vorhandene Middlewares durch eine eigene Lösung ersetzt werden 
			müssen, die vorhandene Middleware kann deaktiviert werden, indem man in der Datei
			folgenden Code hinzufügt:

			\begin{lstlisting}
				return [
				  'frontend' => [
				    'middleware-identifier' => [
				      'disabled' => true,
				    ],
				  ],
				];
			\end{lstlisting}

		\item Lesen Sie mehr über \href{https://new.typo3.org/community/teams/typo3-development/initiatives/initiative-psr-15/}{PSR-15 Initiative} 

	\end{itemize}

\end{frame}

% ------------------------------------------------------------------------------
% LTXE-SLIDE-START
% LTXE-SLIDE-UID:		c7b4f7f2-a11e6fc7-4ffe6e04-92cc98aa
% LTXE-SLIDE-TITLE:		Extended PSR-7 Requests with TYPO3 Normalized Server Parameters
% LTXE-SLIDE-REFERENCE:	Feature-83736-ExtendedPSR-7RequestsWithTYPO3ServerParameters
% ------------------------------------------------------------------------------

\begin{frame}[fragile]
	\frametitle{Änderungen für Entwickler}
	\framesubtitle{PSR-7 Serveranforderungen}

	% decrease font size for code listing
	\lstset{basicstyle=\tiny\ttfamily}

	\begin{itemize}

		\item PSR-7-basierte ServerRequest-Objekte enthalten ein TYPO3-spezifisches 
			Attributobjekt für normalisierte Serverparameter
		\item Das Objekt ist \textbf{momentan} als Attribut des 
			\texttt{ServerRequestInterface \$request} Objekte verfügbar.

			\begin{lstlisting}
				/** @var NormalizedParams $normalizedParams */
				$normalizedParams = $request->getAttribute('normalizedParams');
				$requestPort = $normalizedParams->getRequestPort();
			\end{lstlisting}

		\item Dies ersetzt \texttt{GeneralUtility::getIndpEnv()} und Behauptungen
			wie zum Beispiel \texttt{SCRIPT\_NAME}, \texttt{REQUEST\_URI}, usw können ersetzt werden
			\newline
			(siehe \href{https://docs.typo3.org/typo3cms/extensions/core/latest/Changelog/9.2/Feature-83736-ExtendedPSR-7RequestsWithTYPO3ServerParameters.html}{Dokumentation} für mehrere Informationen)

	\end{itemize}

\end{frame}

% ------------------------------------------------------------------------------
% LTXE-SLIDE-START
% LTXE-SLIDE-UID:		c7b4f7f2-a11e6fc7-4ffe6e04-92cc98aa
% LTXE-SLIDE-TITLE:		PSR-7 and PSR-15 Related Changes
% LTXE-SLIDE-REFERENCE:	Important-83724-APIAndBehaviorChangeInRequestHandlerClasses
% ------------------------------------------------------------------------------

\begin{frame}[fragile]
	\frametitle{Änderungen für Entwickler}
	\framesubtitle{Änderungen die zur PSR-7 and PSR-15 gehören}

	% decrease font size for code listing
	\lstset{basicstyle=\tiny\ttfamily}

	\begin{itemize}
		\item Die internen Request-Handler-Klassen wurden geändert:

		\begin{itemize}
			\item Alle Methoden haben strenge Argumente und Rückgabetypdeklarationen erhalten
			\item Anstatt \texttt{HttpUtility::redirect()},\newline
				ein \texttt{RedirectResponse} wird zurückgegeben
			\item Anstatt Null wird eine NullResponse zurückgegeben
		\end{itemize}

	\end{itemize}

\end{frame}

% ------------------------------------------------------------------------------
% LTXE-SLIDE-START
% LTXE-SLIDE-UID:		c7b4f7f2-a11e6fc7-4ffe6e04-92cc98aa
% LTXE-SLIDE-TITLE:		Environment Class
% LTXE-SLIDE-REFERENCE:	Feature-84153-IntroduceAGenericEnvironmentClass
% ------------------------------------------------------------------------------

\begin{frame}[fragile]
	\frametitle{Änderungen für Entwickler}
	\framesubtitle{Enwironment-Klasse}

	% decrease font size for code listing
	\lstset{basicstyle=\tiny\ttfamily}

	\begin{itemize}
		\item Die neue Basis-API-Klasse stellt anwendungsübergreifende Informationen zu Pfaden
			und PHP-internals, auf die bisher über PHP Konstanten zugegriffen werden konnte:
				\small
					\texttt{TYPO3\textbackslash
						CMS\textbackslash
						Core\textbackslash
						Core\textbackslash
						Environment}
				\normalsize

		\item Die folgenden statischen API-Methoden sind verfügbar:

			\begin{itemize}
				\smaller
				\item \texttt{Environment::isCli()}
					% defines whether TYPO3 runs on a CLI context or HTTP context
				\item \texttt{Environment::getApplicationContext()}
					% returns the ApplicationContext object that encapsulates TYPO3_CONTEXT
				\item \texttt{Environment::isComposerMode()}
					% defines whether TYPO3 was installed via composer
				\item \texttt{Environment::getProjectPath()}
					% returns the absolute path to the root-level folder without the trailing slash
				\item \texttt{Environment::getPublicPath()}
					% returns the absolute path to the publically accessible folder (previously known as PATH_site) without the trailing slash
				\item \texttt{Environment::getVarPath()}
					% returns the absolute path to the folder where non-public semi-persistent files can be stored. For regular projects, this is known as PATH_site/typo3temp/var
				\item \texttt{Environment::getConfigPath()}
					% returns the absolute path to the folder where (writeable) configuration is stored. For regular projects, this is known as PATH_site/typo3conf
				\item \texttt{Environment::getCurrentScript()}
					% the absolute path and filename to the currently executed PHP script
				\item \texttt{Environment::isWindows()}
					% whether TYPO3 runs on a windows server
				\item \texttt{Environment::isUnix()}
					% whether TYPO3 runs on a unix server
			\end{itemize}

	\end{itemize}

\end{frame}

% ------------------------------------------------------------------------------
% LTXE-SLIDE-START
% LTXE-SLIDE-UID:		c7b4f7f2-a11e6fc7-4ffe6e04-92cc98aa
% LTXE-SLIDE-TITLE:		Add recursive filtering of arrays
% LTXE-SLIDE-REFERENCE:	83350-AddRecursiveArrayFiltering
% ------------------------------------------------------------------------------

\begin{frame}[fragile]
	\frametitle{Änderungen für Entwickler}
	\framesubtitle{String Constraints suchen}

	% decrease font size for code listing
	\lstset{basicstyle=\tiny\ttfamily}

	\begin{itemize}
		\item Ein neuer Hook ermöglicht die Änderung von Suchtechnischen Einschränkungen:

			\begin{lstlisting}
				// EXT:my_site/ext_localconf.php
				$dbRecordList = \TYPO3\CMS\Recordlist\RecordList\DatabaseRecordList::class;
				$GLOBALS['TYPO3_CONF_VARS']['SC_OPTIONS'][$dbRecordList]['makeSearchStringConstraints'][123] =
				  \MyVendor\MySite\Hooks\DatabaseRecordListHook::class . '->makeSearchStringConstraints';
			\end{lstlisting}

			\begin{lstlisting}
				// EXT:my_site/Classes/Hooks/DatabaseRecordListHook.php
				namespace MyVendor\MySite\Hooks;
				class DatabaseRecordListHook
				{
				  public function makeSearchStringConstraints(
				    \TYPO3\CMS\Core\Database\Query\QueryBuilder $queryBuilder
				    array $constraints,
				    string $searchString,
				    string $table,
				    int $currentPid,
				  ) {
				    return $constraints;
				  }
				}
			\end{lstlisting}

	\end{itemize}

\end{frame}

% ------------------------------------------------------------------------------
% LTXE-SLIDE-START
% LTXE-SLIDE-UID:		c7b4f7f2-a11e6fc7-4ffe6e04-92cc98aa
% LTXE-SLIDE-TITLE:		Signal/Slot for User Switch
% LTXE-SLIDE-REFERENCE:	Feature-80263-AddANewSignalSlotForUserSwitch
% ------------------------------------------------------------------------------

\begin{frame}[fragile]
	\frametitle{Änderungen für Entwickler}
	\framesubtitle{Signal/Slot für User Switch}

	% decrease font size for code listing
	\lstset{basicstyle=\tiny\ttfamily}

	\begin{itemize}
		\item Ein neues Signal wird ausgegeben, wenn ein Admin-Benutzer im TYPO3-Backend zu einem
			anderen Benutzer wechselt

			\begin{lstlisting}
				$dispatcher = \TYPO3\CMS\Core\Utility\GeneralUtility::makeInstance(
				  \TYPO3\CMS\Extbase\SignalSlot\Dispatcher::class
				);

				$dispatcher->connect(
				  \TYPO3\CMS\Beuser\Controller\BackendUserController::class,
				  'switchUser',
				  \MyVendor\MyExtension\Slots\BackendUserController::class,
				  'switchUser'
				);
			\end{lstlisting}

	\end{itemize}

\end{frame}

% ------------------------------------------------------------------------------
% LTXE-SLIDE-START
% LTXE-SLIDE-UID:		c7b4f7f2-a11e6fc7-4ffe6e04-92cc98aa
% LTXE-SLIDE-TITLE:		ViewHelper Changes (1)
% LTXE-SLIDE-REFERENCE:	Feature-82704-AddReadonlyAndRequiredAttributesToTextareaViewHelper
% ------------------------------------------------------------------------------

\begin{frame}[fragile]
	\frametitle{Änderungen für Entwickler}
	\framesubtitle{ViewHelper Änderungen (1)}

	% decrease font size for code listing
	\lstset{basicstyle=\tiny\ttfamily}

	\begin{itemize}
		\item ViewHelper \texttt{f:form.textarea} unterstützt zwei neue Attribute\newline
			 "\texttt{readonly}" and "\texttt{required}"

			\begin{lstlisting}
				<!-- Set required attribute -->
				<f:form.textarea name="foobar" required="1" />

				<!-- Set readonly attribute -->
				<f:form.textarea name="foobar" readonly="1" />
			\end{lstlisting}

		\item ViewHelpers \texttt{f:uri.typolink} und \texttt{f:uri.typolink} unterstützen das neue
			Attribut "\texttt{absolute}"

			\begin{lstlisting}
				<f:link.typolink parameter="23" absolute="true">Link</f:link.typolink>
				<f:uri.typolink parameter="23" absolute="true" />
			\end{lstlisting}

		\item ViewHelper \texttt{f:render} unterstützt das neue Attribut "\texttt{debug}"
			das ermöglicht die Debug-Ausgabe in einigen Spezielfällen zu deaktivieren

	\end{itemize}

\end{frame}

% ------------------------------------------------------------------------------
% LTXE-SLIDE-START
% LTXE-SLIDE-UID:		c7b4f7f2-a11e6fc7-4ffe6e04-92cc98aa
% LTXE-SLIDE-TITLE:		ViewHelper Changes (2)
% LTXE-SLIDE-REFERENCE:	Feature-83942-ProvideViewHelperToRenderIconForResources
% ------------------------------------------------------------------------------

\begin{frame}[fragile]
	\frametitle{Änderungen für Entwickler}
	\framesubtitle{ViewHelper Änderungen (2)}

	% decrease font size for code listing
	\lstset{basicstyle=\tiny\ttfamily}

	\begin{itemize}
		\item Der neue ViewHelper gibt das Icon-Markup wieder basierend auf einer FAL-Resource

%			\begin{lstlisting}
%				<core:iconForResource resource="{file}" />
%			\end{lstlisting}

			\smaller
				\texttt{<core:iconForResource resource="\{file\}" />}
			\normalsize

	\end{itemize}

\end{frame}

% ------------------------------------------------------------------------------
% LTXE-SLIDE-START
% LTXE-SLIDE-UID:		c7b4f7f2-a11e6fc7-4ffe6e04-92cc98aa
% LTXE-SLIDE-TITLE:		Admin Panel Customization
% LTXE-SLIDE-REFERENCE:	Feature-84045-NewAdminPanelModuleAPI
% ------------------------------------------------------------------------------

\begin{frame}[fragile,label=AdminPanelCustomization]
	\frametitle{Änderungen für Entwickler}
	\framesubtitle{Admin Panel Anpassung}

	% decrease font size for code listing
	\lstset{basicstyle=\tiny\ttfamily}

	\begin{itemize}
		\item Das Admin Panel kann durch benutzerdefinierte Module erweitert werden:
		\item Modulregistrierungsbeispiel:

			\begin{lstlisting}
				$GLOBALS['TYPO3_CONF_VARS']['EXTCONF']['adminpanel']['modules']['yourmodulename'] = [
				  'module' => \MyVendor\Package\AdminPanel\YourModule::class,
				  'after' => ['preview']
				]
			\end{lstlisting}

	\end{itemize}

\end{frame}

% ------------------------------------------------------------------------------
