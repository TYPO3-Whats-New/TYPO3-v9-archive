% ------------------------------------------------------------------------------
% TYPO3 Version 9.3 - What's New - Chapter "Changes for Developers" (English Version)
%
% @author	Michael Schams <schams.net>
% @license	Creative Commons BY-NC-SA 3.0
% @link		http://typo3.org/download/release-notes/whats-new/
% @language	English
% ------------------------------------------------------------------------------
% LTXE-CHAPTER-UID:		846d40ce-66fdc2ec-750dcf95-33ce93e0
% LTXE-CHAPTER-NAME:	Changes for Developers
% ------------------------------------------------------------------------------

\section{Changes for Developers}
\begin{frame}[fragile]
	\frametitle{Changes for Developers}

	\begin{center}\huge{Chapter 3:}\end{center}
	\begin{center}\huge{\color{typo3darkgrey}\textbf{Changes for Developers}}\end{center}

\end{frame}

% ------------------------------------------------------------------------------
% LTXE-SLIDE-START
% LTXE-SLIDE-UID:		f38ee3cb-51d1aa0a-4f6884bf-adadc9e3
% LTXE-SLIDE-TITLE:		Management Database Columns
% LTXE-SLIDE-REFERENCE:	#85160 - Auto create management DB fields from TCA ctrl
% ------------------------------------------------------------------------------

\begin{frame}[fragile]
	\frametitle{Changes for Developers}
	\framesubtitle{"Management" Database Columns}

	\begin{itemize}
		\item Database schema analyzer automatically creates TYPO3 "management"
			columns by reading the TCA
		\item Developers do not need to state these fields in file
			\texttt{ext\_tables.sql}
		\item Management fields are for example:\newline
			\texttt{uid}, \texttt{pid}, \texttt{crdate}, \texttt{cruser},
			\texttt{hidden}, \texttt{deleted}, \texttt{sortby}, etc.
		\item Field definitions in \texttt{ext\_tables.sql} take precedence
			over automatically generated fields, which means they can be
			customized if required
	\end{itemize}

\end{frame}

% ------------------------------------------------------------------------------
% LTXE-SLIDE-START
% LTXE-SLIDE-UID:		f38ee3cb-51d1aa0a-4f6884bf-adadc9e3
% LTXE-SLIDE-TITLE:		Meta Tag Manager (API) (1)
% LTXE-SLIDE-REFERENCE:	#81464 - Add API for meta tag management
% ------------------------------------------------------------------------------

\begin{frame}[fragile]
	\frametitle{Changes for Developers}
	\framesubtitle{Meta Tag Manager (1)}

	% decrease font size for code listing
	\lstset{basicstyle=\tiny\ttfamily}

	\begin{itemize}
		\item New \texttt{MetaTagManager} API has been introduced to manage and
			render meta tags in a flexible, but regulated way
		\item TYPO3 core ships an \href{http://ogp.me/}{Open Graph}
			\texttt{MetaTagManager} for example

			\begin{lstlisting}
				use \TYPO3\CMS\Core\MetaTag\MetaTagManagerRegistry;
				$metaTagManager = MetaTagManagerRegistry::getInstance()->getManagerForProperty('og:title');
				$metaTagManager->addProperty('og:title', 'This is the OG title from a controller');
			\end{lstlisting}

		\item Example functions available:

			\begin{itemize}
				\smaller
					\item \texttt{\$metaTagManager->addProperty()}
					\item \texttt{\$metaTagManager->removeProperty()}
					\item \texttt{\$metaTagManager->removeAllProperties()}
			\end{itemize}

	\end{itemize}

\end{frame}

% ------------------------------------------------------------------------------
% LTXE-SLIDE-START
% LTXE-SLIDE-UID:		f38ee3cb-51d1aa0a-4f6884bf-adadc9e3
% LTXE-SLIDE-TITLE:		Meta Tag Manager (API) (2)
% LTXE-SLIDE-REFERENCE:	#81464 - Add API for meta tag management
% ------------------------------------------------------------------------------

\begin{frame}[fragile]
	\frametitle{Changes for Developers}
	\framesubtitle{Meta Tag Manager (2)}

	% decrease font size for code listing
	\lstset{basicstyle=\tiny\ttfamily}

	\begin{itemize}
		\item Developers can register custom \texttt{MetaTagManager} in the
			\texttt{MetaTagManagerRegistry}

			\begin{lstlisting}
				use \TYPO3\CMS\Core\MetaTag\MetaTagManagerRegistry;
				$metaTagManagerRegistry = MetaTagManagerRegistry::getInstance();
				$metaTagManagerRegistry->registerManager(
				  'custom',
				  \Some\CustomExtension\MetaTag\CustomMetaTagManager::class
				);
			\end{lstlisting}

		\item Meta tags can be set by TypoScript and PHP

			\begin{lstlisting}
				page.meta {
				  og:site_name = TYPO3
				  og:site_name.attribute = property
				  og:site_name.replace = 1
				}
			\end{lstlisting}

			\smaller
				("\texttt{replace = 1}" replaces earlier set meta tags)
			\normalsize

	\end{itemize}

\end{frame}

% ------------------------------------------------------------------------------
% LTXE-SLIDE-START
% LTXE-SLIDE-UID:		f38ee3cb-51d1aa0a-4f6884bf-adadc9e3
% LTXE-SLIDE-TITLE:		Doctrine: Negative DateInterval Fields
% LTXE-SLIDE-REFERENCE:	#84744 - Raise doctrine/dbal-version
% ------------------------------------------------------------------------------

\begin{frame}[fragile]
	\frametitle{Changes for Developers}
	\framesubtitle{Doctrine: Negative DateInterval Fields}

	\begin{itemize}
		\item Database abstraction layer "Doctrine" has been updated to version 2.7.1
		\item Format of \texttt{DateInterval} fields can now be negative, which
			means, they must begin with either "\texttt{+}" or "\texttt{-}"
		\item Migration: Assuming, negative \texttt{DateIntervals} have not been used yet,
			simply prefix the data with "\texttt{+}"
	\end{itemize}

	\breakingchange

\end{frame}

% ------------------------------------------------------------------------------
% LTXE-SLIDE-START
% LTXE-SLIDE-UID:		f38ee3cb-51d1aa0a-4f6884bf-adadc9e3
% LTXE-SLIDE-TITLE:		Validate annotation
% LTXE-SLIDE-REFERENCE:	#83167 - Replace @validate with @TYPO3\CMS\Extbase\Annotation\Validate
% ------------------------------------------------------------------------------

\begin{frame}[fragile]
	\frametitle{Changes for Developers}
	\framesubtitle{Validate Annotation as Class Name}

	% decrease font size for code listing
	\lstset{basicstyle=\tiny\ttfamily}

	\begin{itemize}
		\item Doctrine annotation
			\texttt{TYPO3\textbackslash
				CMS\textbackslash
				Extbase\textbackslash
				Annotation\textbackslash
				Validate} has been introduced
		\item This is the successor to the \texttt{\@validate} annotation
		\item Example:

			\begin{lstlisting}
				/**
				 * @TYPO3\CMS\Extbase\Annotation\Validate
				 * @var Foo
				 */
				public $property;
			\end{lstlisting}

		\item The \texttt{use}-statement can also be used:

			\begin{lstlisting}
				use TYPO3\CMS\Extbase\Annotation\Validate;

				/**
				 * @Validate
				 * @var Foo
				 */
				public $property;
			\end{lstlisting}

	\end{itemize}

\end{frame}

% ------------------------------------------------------------------------------
% LTXE-SLIDE-START
% LTXE-SLIDE-UID:		f38ee3cb-51d1aa0a-4f6884bf-adadc9e3
% LTXE-SLIDE-TITLE:		Backend ViewHelpers
% LTXE-SLIDE-REFERENCE:	#83983 - Improved ModuleLinkViewHelper
% LTXE-SLIDE-REFERENCE:	#84983 - BE ViewHelper for EditDocumentController
% ------------------------------------------------------------------------------

\begin{frame}[fragile]
	\frametitle{Changes for Developers}
	\framesubtitle{Backend ViewHelpers}

	% decrease font size for code listing
	\lstset{basicstyle=\fontsize{7.5pt}{10}\selectfont\ttfamily}

	\begin{itemize}
		\item Module Link ViewHelper supports two new arguments\newline
			\small
				\texttt{TYPO3\textbackslash
					CMS\textbackslash
					Backend\textbackslash
					ViewHelpers\textbackslash
					ModuleLinkViewHelper}
			\normalsize

			\begin{itemize}
				\small
				\item \texttt{query}: allow defining query parameters also as string
				\item \texttt{currentUrlParameterName}: argument uses current URL
			\end{itemize}

			This change encourages developers to migrate existing custom backend route
			ViewHelpers to this ViewHelper.

		\item New ViewHelpers for the backend to simplify create/edit records:

			\begin{lstlisting}
				<be:uri.newRecord pid=" ... " table=" ... " />
				<be:link.newRecord pid=" ... " table=" ... " />
				<be:uri.editRecord uid=" ... " table=" ... " />
				<be:link.editRecord uid=" ... " table=" ... " />
			\end{lstlisting}

	\end{itemize}

\end{frame}

% ------------------------------------------------------------------------------
% LTXE-SLIDE-START
% LTXE-SLIDE-UID:		f38ee3cb-51d1aa0a-4f6884bf-adadc9e3
% LTXE-SLIDE-TITLE:		LanguageMenu Processor (1)
% LTXE-SLIDE-REFERENCE:	#84650 - Introduce fluid data processor for language menus
% ------------------------------------------------------------------------------

\begin{frame}[fragile]
	\frametitle{Changes for Developers}
	\framesubtitle{LanguageMenu Processor (1)}

	% decrease font size for code listing
	\lstset{basicstyle=\tiny\ttfamily}

	\begin{itemize}
		\item New \texttt{LanguageMenuProcessor} for Fluid has been introduced

			\begin{lstlisting}
				10 = TYPO3\CMS\Frontend\DataProcessing\LanguageMenuProcessor
				10 {
				  languages = auto
				  as = languageNavigation
				}
			\end{lstlisting}

		\item Options:

			\begin{itemize}

				\item \texttt{if}: TypoScript \texttt{if} condition
				\item \texttt{languages}: list of comma separated language IDs
					(e.g. 0,1,2) "\texttt{auto}" to load from site languages
				\item \texttt{as}: Variable to be used within the result
			\end{itemize}

	\end{itemize}

\end{frame}

% ------------------------------------------------------------------------------
% LTXE-SLIDE-START
% LTXE-SLIDE-UID:		f38ee3cb-51d1aa0a-4f6884bf-adadc9e3
% LTXE-SLIDE-TITLE:		LanguageMenu Processor (2)
% LTXE-SLIDE-REFERENCE:	#84650 - Introduce fluid data processor for language menus
% ------------------------------------------------------------------------------

\begin{frame}[fragile]
	\frametitle{Changes for Developers}
	\framesubtitle{LanguageMenu Processor (2)}

	% decrease font size for code listing
	\lstset{basicstyle=\tiny\ttfamily}

	\begin{itemize}
		\item Example Fluid-Template:

			\begin{lstlisting}
				<f:if condition="{languageNavigation}">
				  <ul id="language" class="language-menu">
				    <f:for each="{languageNavigation}" as="item">
				      <li class="{f:if(condition: item.active, then: 'active')}{f:if(condition: item.available, else: ' text-muted')}">
				        <f:if condition="{item.available}">
				          <f:then>
				            <a href="{item.link}" hreflang="{item.hreflang}" title="{item.navigationTitle}">
				              <span>{item.navigationTitle}</span>
				            </a>
				          </f:then>
				          <f:else>
				            <span>{item.navigationTitle}</span>
				          </f:else>
				        </f:if>
				      </li>
				    </f:for>
				  </ul>
				</f:if>
			\end{lstlisting}

	\end{itemize}

\end{frame}

% ------------------------------------------------------------------------------
% LTXE-SLIDE-START
% LTXE-SLIDE-UID:		f38ee3cb-51d1aa0a-4f6884bf-adadc9e3
% LTXE-SLIDE-TITLE:		Miscellaneous (1)
% LTXE-SLIDE-REFERENCE:	#85025 - Enumerations are now final
% LTXE-SLIDE-REFERENCE:	#84244 - Allow adding additional query restrictions
% ------------------------------------------------------------------------------

\begin{frame}[fragile]
	\frametitle{Changes for Developers}
	\framesubtitle{Miscellaneous}

	\begin{itemize}
		\item All enumeration classes in TYPO3 have been marked as "\texttt{final}"
			and 3rd party classes extending enumerations triggers a fatal PHP error
		\item Additional query restrictions can be added to:\newline
			\smaller
				\texttt{\$GLOBALS['TYPO3\_CONF\_VARS']['DB']['additionalQueryRestrictions']}
			\normalsize\newline
			These restriction objects will be added to any select query executed
			using the QueryBuilder (use with caution!)

	\end{itemize}

\end{frame}

% ------------------------------------------------------------------------------
