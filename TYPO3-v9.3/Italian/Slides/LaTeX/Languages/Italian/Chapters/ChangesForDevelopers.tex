% ------------------------------------------------------------------------------
% TYPO3 Version 9.3 - What's New - Chapter "Changes for Developers" (English Version)
%
% @author	Michael Schams <schams.net>
% @license	Creative Commons BY-NC-SA 3.0
% @link		http://typo3.org/download/release-notes/whats-new/
% @language	English
% ------------------------------------------------------------------------------
% LTXE-CHAPTER-UID:		846d40ce-66fdc2ec-750dcf95-33ce93e0
% LTXE-CHAPTER-NAME:	Changes for Developers
% ------------------------------------------------------------------------------

\section{Modifiche per sviluppatori}
\begin{frame}[fragile]
	\frametitle{Modifiche per sviluppatori}

	\begin{center}\huge{Capitolo 3:}\end{center}
	\begin{center}\huge{\color{typo3darkgrey}\textbf{Modifiche per sviluppatori}}\end{center}

\end{frame}

% ------------------------------------------------------------------------------
% LTXE-SLIDE-START
% LTXE-SLIDE-UID:		03758f49-7d13a084-13656ffa-71661967
% LTXE-SLIDE-TITLE:		Management Database Columns
% LTXE-SLIDE-REFERENCE:	#85160 - Auto create management DB fields from TCA ctrl
% ------------------------------------------------------------------------------

\begin{frame}[fragile]
	\frametitle{Modifiche per sviluppatori}
	\framesubtitle{"Gestione" delle colonne del Database}

	\begin{itemize}
		\item L'analizzatore della struttura del database crea automaticamente la "gestione" 
			delle colonne leggendo il TCA
		\item Gli sviluppatori non hanno bisogno di dichiarare alcuni campi nel file
			\texttt{ext\_tables.sql}
		\item Campi gestiti sono per esempio:\newline
			\texttt{uid}, \texttt{pid}, \texttt{crdate}, \texttt{cruser},
			\texttt{hidden}, \texttt{deleted}, \texttt{sortby}, ecc.
		\item Le definizioni dei campi in \texttt{ext\_tables.sql} hanno la precedenza
			sui campi generati automaticamente, il che significa che possono essere personalizzati
			se necessario
	\end{itemize}

\end{frame}

% ------------------------------------------------------------------------------
% LTXE-SLIDE-START
% LTXE-SLIDE-UID:		528b70af-92597ea1-d9644e10-33852a00
% LTXE-SLIDE-TITLE:		Meta Tag Manager (API) (1)
% LTXE-SLIDE-REFERENCE:	#81464 - Add API for meta tag management
% ------------------------------------------------------------------------------

\begin{frame}[fragile]
	\frametitle{Modifiche per sviluppatori}
	\framesubtitle{Gestore Meta Tag (1)}

	% decrease font size for code listing
	\lstset{basicstyle=\tiny\ttfamily}

	\begin{itemize}
		\item Le nuove API \texttt{MetaTagManager} sono state introdotte per gestire e
			renderizzare i meta tag in modo flessibile, ma regolamentato
		\item il core TYPO3 genera un \href{http://ogp.me/}{Open Graph}
			\texttt{MetaTagManager} per esempio

			\begin{lstlisting}
				use \TYPO3\CMS\Core\MetaTag\MetaTagManagerRegistry;
				$metaTagManager = MetaTagManagerRegistry::getInstance()->getManagerForProperty('og:title');
				$metaTagManager->addProperty('og:title', 'This is the OG title from a controller');
			\end{lstlisting}

		\item Funzioni di esempio disponibili:

			\begin{itemize}
				\smaller
					\item \texttt{\$metaTagManager->addProperty()}
					\item \texttt{\$metaTagManager->removeProperty()}
					\item \texttt{\$metaTagManager->removeAllProperties()}
			\end{itemize}

	\end{itemize}

\end{frame}

% ------------------------------------------------------------------------------
% LTXE-SLIDE-START
% LTXE-SLIDE-UID:		647f1b2d-e07d8733-5d45d1d7-22e31292
% LTXE-SLIDE-TITLE:		Meta Tag Manager (API) (2)
% LTXE-SLIDE-REFERENCE:	#81464 - Add API for meta tag management
% ------------------------------------------------------------------------------

\begin{frame}[fragile]
	\frametitle{Modifiche per sviluppatori}
	\framesubtitle{Gestore Meta Tag (2)}

	% decrease font size for code listing
	\lstset{basicstyle=\tiny\ttfamily}

	\begin{itemize}
		\item Gli sviluppatori possono registrare \texttt{MetaTagManager} personalizzati nel
			\texttt{MetaTagManagerRegistry}

			\begin{lstlisting}
				use \TYPO3\CMS\Core\MetaTag\MetaTagManagerRegistry;
				$metaTagManagerRegistry = MetaTagManagerRegistry::getInstance();
				$metaTagManagerRegistry->registerManager(
				  'custom',
				  \Some\CustomExtension\MetaTag\CustomMetaTagManager::class
				);
			\end{lstlisting}

		\item I meta tag possono essere impostati via TypoScript e PHP

			\begin{lstlisting}
				page.meta {
				  og:site_name = TYPO3
				  og:site_name.attribute = property
				  og:site_name.replace = 1
				}
			\end{lstlisting}

			\smaller
				("\texttt{replace = 1}" replaces earlier set meta tags)
			\normalsize

	\end{itemize}

\end{frame}

% ------------------------------------------------------------------------------
% LTXE-SLIDE-START
% LTXE-SLIDE-UID:		296ad890-d3b286c8-e9310a46-9dca13a3
% LTXE-SLIDE-TITLE:		Doctrine: Negative DateInterval Fields
% LTXE-SLIDE-REFERENCE:	#84744 - Raise doctrine/dbal-version
% ------------------------------------------------------------------------------

\begin{frame}[fragile]
	\frametitle{Modifiche per sviluppatori}
	\framesubtitle{Doctrine: Campi intervallo di date negativi}

	\begin{itemize}
		\item Lo strato di astrazione del database "Doctrine" è stato aggiornato alla versione 2.7.1
		\item Il formato dei campi \texttt{DateInterval} possono essere negativi, questo significa,
			che devono iniziare con "\texttt{+}" o "\texttt{-}"
		\item Migrazione: Supponendo che \texttt{DateIntervals} negativi non sono stati utilizzati,
			semplicemente anteponi alla data il segno "\texttt{+}"
	\end{itemize}

	\breakingchange

\end{frame}

% ------------------------------------------------------------------------------
% LTXE-SLIDE-START
% LTXE-SLIDE-UID:		25707e2b-fde73aaa-020a60ca-9e6f0cb8
% LTXE-SLIDE-TITLE:		Validate annotation
% LTXE-SLIDE-REFERENCE:	#83167 - Replace @validate with @TYPO3\CMS\Extbase\Annotation\Validate
% ------------------------------------------------------------------------------

\begin{frame}[fragile]
	\frametitle{Modifiche per sviluppatori}
	\framesubtitle{Convalida dell'annotazione come nome di classe}

	% decrease font size for code listing
	\lstset{basicstyle=\tiny\ttfamily}

	\begin{itemize}
		\item Annotazione Doctrine
			\texttt{TYPO3\textbackslash
				CMS\textbackslash
				Extbase\textbackslash
				Annotation\textbackslash
				Validate} è stata introdotta
		\item Questo è il successore dell'annotazione \texttt{\@validate}
		\item Esempio:

			\begin{lstlisting}
				/**
				 * @TYPO3\CMS\Extbase\Annotation\Validate
				 * @var Foo
				 */
				public $property;
			\end{lstlisting}

		\item La dichiarazione \texttt{use} può anch'essa essere utilizzata:

			\begin{lstlisting}
				use TYPO3\CMS\Extbase\Annotation\Validate;

				/**
				 * @Validate
				 * @var Foo
				 */
				public $property;
			\end{lstlisting}

	\end{itemize}

\end{frame}

% ------------------------------------------------------------------------------
% LTXE-SLIDE-START
% LTXE-SLIDE-UID:		31e5ebab-05661f21-0e6a6c3a-35413d7c
% LTXE-SLIDE-TITLE:		Backend ViewHelpers
% LTXE-SLIDE-REFERENCE:	#83983 - Improved ModuleLinkViewHelper
% LTXE-SLIDE-REFERENCE:	#84983 - BE ViewHelper for EditDocumentController
% ------------------------------------------------------------------------------

\begin{frame}[fragile]
	\frametitle{Modifiche per sviluppatori}
	\framesubtitle{ViewHelper di Backend}

	% decrease font size for code listing
	\lstset{basicstyle=\fontsize{7.5pt}{10}\selectfont\ttfamily}

	\begin{itemize}
		\item Il modulo ViewHelper Link supporta ora due nuovi parametri\newline
			\small
				\texttt{TYPO3\textbackslash
					CMS\textbackslash
					Backend\textbackslash
					ViewHelpers\textbackslash
					ModuleLinkViewHelper}
			\normalsize

			\begin{itemize}
				\small
				\item \texttt{query}: permette la definizione dei parametri di query anche come stringa
				\item \texttt{currentUrlParameterName}: il parametro utilizza l'URL corrente
			\end{itemize}

			Questo cambiamento incoraggia gli sviluppatori a migrare ViewHelper personalizzati
			a questo ViewHelper.

		\item Nuovi ViewHelper per il backend a semplificazione della creazione/modifica di record:

			\begin{lstlisting}
				<be:uri.newRecord pid=" ... " table=" ... " />
				<be:link.newRecord pid=" ... " table=" ... " />
				<be:uri.editRecord uid=" ... " table=" ... " />
				<be:link.editRecord uid=" ... " table=" ... " />
			\end{lstlisting}

	\end{itemize}

\end{frame}

% ------------------------------------------------------------------------------
% LTXE-SLIDE-START
% LTXE-SLIDE-UID:		c64da4b1-97e71f88-eb1117cb-8eb4dc9d
% LTXE-SLIDE-TITLE:		LanguageMenu Processor (1)
% LTXE-SLIDE-REFERENCE:	#84650 - Introduce fluid data processor for language menus
% ------------------------------------------------------------------------------

\begin{frame}[fragile]
	\frametitle{Modifiche per sviluppatori}
	\framesubtitle{LanguageMenu Processor (1)}

	% decrease font size for code listing
	\lstset{basicstyle=\tiny\ttfamily}

	\begin{itemize}
		\item Sono stati introdotti nuovi \texttt{LanguageMenuProcessor} per Fluid:

			\begin{lstlisting}
				10 = TYPO3\CMS\Frontend\DataProcessing\LanguageMenuProcessor
				10 {
				  languages = auto
				  as = languageNavigation
				}
			\end{lstlisting}

		\item Opzioni:

			\begin{itemize}

				\item \texttt{if}: condizione TypoScript \texttt{if}
				\item \texttt{languages}: lista degli ID separati da virgola 
					(es. 0,1,2) "\texttt{auto}" per caricare dalle lingue del sito
				\item \texttt{as}: variabile da usare per il risultato
			\end{itemize}

	\end{itemize}

\end{frame}

% ------------------------------------------------------------------------------
% LTXE-SLIDE-START
% LTXE-SLIDE-UID:		33e9e594-b6649517-e08e83f8-1e1101e0
% LTXE-SLIDE-TITLE:		LanguageMenu Processor (2)
% LTXE-SLIDE-REFERENCE:	#84650 - Introduce fluid data processor for language menus
% ------------------------------------------------------------------------------

\begin{frame}[fragile]
	\frametitle{Modifiche per sviluppatori}
	\framesubtitle{LanguageMenu Processor (2)}

	% decrease font size for code listing
	\lstset{basicstyle=\tiny\ttfamily}

	\begin{itemize}
		\item Esempio di Template Fluid:

			\begin{lstlisting}
				<f:if condition="{languageNavigation}">
				  <ul id="language" class="language-menu">
				    <f:for each="{languageNavigation}" as="item">
				      <li class="{f:if(condition: item.active, then: 'active')}{f:if(condition: item.available, else: ' text-muted')}">
				        <f:if condition="{item.available}">
				          <f:then>
				            <a href="{item.link}" hreflang="{item.hreflang}" title="{item.navigationTitle}">
				              <span>{item.navigationTitle}</span>
				            </a>
				          </f:then>
				          <f:else>
				            <span>{item.navigationTitle}</span>
				          </f:else>
				        </f:if>
				      </li>
				    </f:for>
				  </ul>
				</f:if>
			\end{lstlisting}

	\end{itemize}

\end{frame}

% ------------------------------------------------------------------------------
% LTXE-SLIDE-START
% LTXE-SLIDE-UID:		69282159-29a2f9ed-16c4ec47-f8737374
% LTXE-SLIDE-TITLE:		Miscellaneous (1)
% LTXE-SLIDE-REFERENCE:	#85025 - Enumerations are now final
% LTXE-SLIDE-REFERENCE:	#84244 - Allow adding additional query restrictions
% ------------------------------------------------------------------------------

\begin{frame}[fragile]
	\frametitle{Modifiche per sviluppatori}
	\framesubtitle{Varie}

	\begin{itemize}
		\item Tutte le classi di enumerazione in TYPO3 sono state contrassegnate  come "\texttt{final}"
			e le classi di terze parti che estendono le enumerazioni innescano un errore PHP fatale
		\item Ulteriori restrizioni di query possono essere aggiunte a:\newline
			\smaller
				\texttt{\$GLOBALS['TYPO3\_CONF\_VARS']['DB']['additionalQueryRestrictions']}
			\normalsize\newline
			Queste restrizioni saranno aggiunte a qualsiasi query eseguita
			usando QueryBuilder (usa con cautela!)

	\end{itemize}

\end{frame}

% ------------------------------------------------------------------------------
