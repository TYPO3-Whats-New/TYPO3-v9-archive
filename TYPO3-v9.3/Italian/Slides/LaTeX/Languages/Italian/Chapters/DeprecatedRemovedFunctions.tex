% ------------------------------------------------------------------------------
% TYPO3 Version 9.3 - What's New - Chapter "Deprecated Functions" (English Version)
%
% @author	Michael Schams <schams.net>
% @license	Creative Commons BY-NC-SA 3.0
% @link		http://typo3.org/download/release-notes/whats-new/
% @language	English
% ------------------------------------------------------------------------------
% LTXE-CHAPTER-UID:		3f842373-9262b8d3-f9c8de76-cf29ce17
% LTXE-CHAPTER-NAME:	Deprecated Functions
% ------------------------------------------------------------------------------

\section{Funzionalità deprecate/rimosse}
\begin{frame}[fragile]
	\frametitle{Funzionalità deprecate/rimosse}

	\begin{center}\huge{Capitolo 4:}\end{center}
	\begin{center}\huge{\color{typo3darkgrey}\textbf{Funzionalità deprecate/rimosse}}\end{center}

\end{frame}

% ------------------------------------------------------------------------------
% LTXE-SLIDE-START
% LTXE-SLIDE-UID:		61d3c225-ce158e98-d34e2215-bd036699
% LTXE-SLIDE-TITLE:		User Passwords
% LTXE-SLIDE-REFERENCE:	#85026 - salted passwords changes
% LTXE-SLIDE-REFERENCE:	#85027 - Salted passwords related methods and classes
% ------------------------------------------------------------------------------

\begin{frame}[fragile]
	\frametitle{Funzionalità deprecate/rimosse}
	\framesubtitle{Password degli utenti}

	\begin{itemize}
		\item Il task dello scheduler "Convert user passwords to salted hashes"
			è stato \textbf{rimosso}\newline
			\smaller
				(cerca i valori che iniziano per "\texttt{\$}" nelle tabelle
				\texttt{be\_users} e \texttt{fe\_users} per trovare i record degli utenti,
				che sono ancora in chiaro o con MD5 hash)
			\normalsize
		\item La seguente funzione è stata segnata come \textbf{deprecata}:\newline
			\fontsize{7.5pt}{10}\selectfont
				\texttt{TYPO3\textbackslash
					CMS\textbackslash
					saltedpasswords\textbackslash
					Utility\textbackslash
					SaltedPasswordsUtility::isUsageEnabled()}
			\normalsize

	\end{itemize}

\end{frame}

% ------------------------------------------------------------------------------
% LTXE-SLIDE-START
% LTXE-SLIDE-UID:		2e04475c-86755b1f-a1e75749-069be08f
% LTXE-SLIDE-TITLE:		Extension lang removed
% LTXE-SLIDE-REFERENCE:	#84680 - Removed unused locallang files from EXT:lang
% LTXE-SLIDE-REFERENCE:	#84680 - Move last language files away from ext:lang and remove ext:lang completely
% ------------------------------------------------------------------------------

\begin{frame}[fragile]
	\frametitle{Funzionalità deprecate/rimosse}
	\framesubtitle{Rimossa l'estensione \texttt{EXT:lang}}

	% decrease font size for code listing
	\lstset{basicstyle=\tiny\ttfamily}

	\begin{itemize}
		\item I file non usati sono stati rimossi dall'estensione \texttt{EXT:lang}
		\item I riferimenti alle traduzioni in \texttt{EXT:lang} ritornano un valore vuoto
		\item I file delle lingue sono stati spostati nelle rispettive estensioni:
	\end{itemize}

	\begin{lstlisting}
		locallang_alt_intro.xlf => about/Resources/Private/Language/Modules/locallang_alt_intro.xlf
		locallang_alt_doc.xlf => backend/Resources/Private/Language/locallang_alt_doc.xlf
		locallang_login.xlf => backend/Resources/Private/Language/locallang_login.xlf
		locallang_common.xlf => core/Resources/Private/Language/locallang_common.xlf
		locallang_core.xlf => core/Resources/Private/Language/locallang_core.xlf
		locallang_general.xlf => core/Resources/Private/Language/locallang_general.xlf
		locallang_misc.xlf => core/Resources/Private/Language/locallang_misc.xlf
		locallang_mod_web_list.xlf => core/Resources/Private/Language/locallang_mod_web_list.xlf
		locallang_tca.xlf => core/Resources/Private/Language/locallang_tca.xlf
		locallang_tsfe.xlf => core/Resources/Private/Language/locallang_tsfe.xlf
		locallang_wizards.xlf => core/Resources/Private/Language/locallang_wizards.xlf
		locallang_browse_links.xlf => recordlist/Resources/Private/Language/locallang_browse_links.xlf
		locallang_tcemain.xlf => workspaces/Resources/Private/Language/locallang_tcemain.xlf
	\end{lstlisting}

\end{frame}

% ------------------------------------------------------------------------------
% LTXE-SLIDE-START
% LTXE-SLIDE-UID:		ee30f730-59ac1ea3-52c7ec35-453aeefa
% LTXE-SLIDE-TITLE:		TSConfig Related Methods
% LTXE-SLIDE-REFERENCE:	#84993 - Deprecate some TSconfig related methods
% ------------------------------------------------------------------------------

\begin{frame}[fragile]
	\frametitle{Funzionalità deprecate/rimosse}
	\framesubtitle{Metodi relativi a TSConfig}

	% decrease font size for code listing
	\lstset{basicstyle=\tiny\ttfamily}

	\begin{itemize}
		\item Metodi relativi a TSConfig Utenti segnati come \textbf{deprecati}:

			\begin{lstlisting}
				TYPO3\CMS\core\Authentication\BackendUserAuthentication->getTSConfigVal()
				TYPO3\CMS\core\Authentication\BackendUserAuthentication->getTSConfigProp()
			\end{lstlisting}

		\item Modifiche al metodo di autenticazione (non sono più permessi parametri):

			\begin{lstlisting}
				TYPO3\CMS\core\Authentication\BackendUserAuthentication->getTSConfig()
			\end{lstlisting}

		\item Metodi relativi a TSConfig Pagina segnati come \textbf{deprecati}:

			\begin{lstlisting}
				TYPO3\CMS\backend\Utility\BackendUtility::getModTSconfig()
				TYPO3\CMS\backend\Utility\BackendUtility::unsetMenuItems()
				TYPO3\CMS\backend\Tree\View\PagePositionMap->getModConfig()
				TYPO3\CMS\core\DataHandling\DataHandler->getTCEMAIN_TSconfig()
			\end{lstlisting}

		\item Propietà impostate con segnalazione di un \textbf{deprecation error} durante l'accesso:

			\begin{lstlisting}
				TYPO3\CMS\backend\Tree\View\PagePositionMap->getModConfigCache
				TYPO3\CMS\backend\Tree\View\PagePositionMap->modConfigStr
			\end{lstlisting}

	\end{itemize}

\end{frame}


% ------------------------------------------------------------------------------
% LTXE-SLIDE-START
% LTXE-SLIDE-UID:		15514e33-6cd8b4a5-259cd891-721639cb
% LTXE-SLIDE-TITLE:		Overriding Page TSConfig
% LTXE-SLIDE-REFERENCE:	#84982 - Overriding page TSconfig mod. with user TSconfig mod.
% ------------------------------------------------------------------------------

\begin{frame}[fragile]
	\frametitle{Funzionalità deprecate/rimosse}
	\framesubtitle{Sovrascrivere TSConfig Pagina}

	% decrease font size for code listing
	\lstset{basicstyle=\smaller\ttfamily}

	\begin{itemize}
		\item Istruzioni TSConfig Utente che iniziano con "\texttt{mod.}" innescano un errore PHP
			\texttt{E\_USER\_DEPRECATED} e termineranno di funzionare in TYPO3 v10
		\item Assicurati di aggiungere il prefisso "\texttt{page.}" all'istruzione
			nel caso un istruzione TSConfig Pagina debba essere sovrascritta in un TSConfig Utente:

			\begin{lstlisting}
				// before
				mod.web_list.disableSingleTableView = 1

				// after
				page.mod.web_list.disableSingleTableView = 1
			\end{lstlisting}

	\end{itemize}

\end{frame}

% ------------------------------------------------------------------------------
% LTXE-SLIDE-START
% LTXE-SLIDE-UID:		fcbb9824-f07a462a-caf69347-e565aa15
% LTXE-SLIDE-TITLE:		URL Handlers
% LTXE-SLIDE-REFERENCE:	#85124 - Redirecting urlHandler Hook Concept
% ------------------------------------------------------------------------------

\begin{frame}[fragile]
	\frametitle{Funzionalità deprecate/rimosse}
	\framesubtitle{Gestore di URL}

	\begin{itemize}
		\item Il concetto di gestore di URL introdotto in TYPO3 v7 per permettere alle pagine
			la redirezione è impostato a \textbf{deprecato} a favore dell'uso dei middleware
			PSR-7/PSR-15 

		\item Chiamando le seguenti funzioni è innescato un errore PHP
			\texttt{E\_USER\_DEPRECATED}:

			\begin{itemize}
				\item \texttt{\$TSFE->initializeRedirectUrlHandlers()}
				\item \texttt{\$TSFE->redirectToExternalUrl()}
			\end{itemize}

	\end{itemize}

\end{frame}

% ------------------------------------------------------------------------------
% LTXE-SLIDE-START
% LTXE-SLIDE-UID:		2f00e55a-684f7c7a-e3b47b6f-cf7a0164
% LTXE-SLIDE-TITLE:		Miscellaneous
% LTXE-SLIDE-REFERENCE:	#81686 - Accessing core TypoScript with .txt file extension has been deprecated
% LTXE-SLIDE-REFERENCE:	#85036 - Removed support for non-namespaced classes in Extbase
% LTXE-SLIDE-REFERENCE:	#85113 - Legacy Backend Module Routing methods
% ------------------------------------------------------------------------------

\begin{frame}[fragile]
	\frametitle{Funzionalità deprecate/rimosse}
	\framesubtitle{Varie}

	\begin{itemize}
		\item I file TypoScript con estensione "\texttt{.txt}" sono stati rinominati
			in "\texttt{.typoscript}" e "\texttt{.tsconfig}"
		\item Le installazioni che includono file di sistema TS che usano la vecchia estensione
			scaturiscono un errore PHP \texttt{E\_USER\_DEPRECATED}
		\item Le classi senza namespaced come 
			"\texttt{Tx\_Extension\_Controller\_FooController}"\newline
			non sono più supportate e non funzionano più
		\item I seguenti metodi sono segnati come \textbf{deprecati}:

			\begin{itemize}
				\item \texttt{BackendUtility::getModuleUrl()}
				\item \texttt{UriBuilder->buildUriFromModule()}
			\end{itemize}

	\end{itemize}

\end{frame}

% ------------------------------------------------------------------------------
% LTXE-SLIDE-START
% LTXE-SLIDE-UID:		e1efedd1-cfb54249-12eb75ef-f53347cd
% LTXE-SLIDE-TITLE:		More functions...
% ------------------------------------------------------------------------------

\begin{frame}[fragile]
	\frametitle{Funzionalità deprecate/rimosse}

	\vspace{0.6cm}
	\begin{center}
		Molte altre funzioni
	\end{center}
	\vspace{-0.8cm}
	\begin{center}
		sono state marcate come deprecate o rimosse
	\end{center}
	\vspace{-0.8cm}
	\begin{center}
		in TYPO3 versione 9.2.
	\end{center}
	\vspace{-0.6cm}
	\begin{center}
		Vedi la \href{https://docs.typo3.org/typo3cms/extensions/core/latest/Changelog/9.3/Index.html#deprecation}{documentazione TYPO3} per altri dettagli.
	\end{center}

\end{frame}

% ------------------------------------------------------------------------------
