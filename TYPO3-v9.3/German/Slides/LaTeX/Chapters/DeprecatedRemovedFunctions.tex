% ------------------------------------------------------------------------------
% TYPO3 Version 9.3 - What's New - Chapter "Deprecated Functions" (German Version)
%
% @author	Michael Schams <schams.net>
% @license	Creative Commons BY-NC-SA 3.0
% @link		http://typo3.org/download/release-notes/whats-new/
% @language	German
% ------------------------------------------------------------------------------
% LTXE-CHAPTER-UID:		3f842373-9262b8d3-f9c8de76-cf29ce17
% LTXE-CHAPTER-NAME:	Deprecated Functions
% ------------------------------------------------------------------------------

\section{Veraltete/Entfernte Funktionen}
\begin{frame}[fragile]
	\frametitle{Veraltete/Entfernte Funktionen}

	\begin{center}\huge{Kapitel 4:}\end{center}
	\begin{center}\huge{\color{typo3darkgrey}\textbf{Veraltete/Entfernte Funktionen}}\end{center}

\end{frame}

% ------------------------------------------------------------------------------
% LTXE-SLIDE-START
% LTXE-SLIDE-UID:		1a5b0bf2-0a03d9a4-25f5eb79-0d3e8d36
% LTXE-SLIDE-TITLE:		User Passwords
% LTXE-SLIDE-REFERENCE:	#85026 - salted passwords changes
% LTXE-SLIDE-REFERENCE:	#85027 - Salted passwords related methods and classes
% ------------------------------------------------------------------------------

\begin{frame}[fragile]
	\frametitle{Veraltete/Entfernte Funktionen}
	\framesubtitle{Benutzerpasswörter}

	\begin{itemize}
		\item Der Scheduler-Task "Convert user passwords to salted hashes"
			wurde \textbf{entfernt}\newline
			\smaller
				(Suche in den Datenbanktabellen n\texttt{be\_users} und \texttt{fe\_users} nach Werten,
				 die mit "\texttt{\$}" beginnen, welche nicht im Klartext
				oder als MD-5 Hash vorliegen)
			\normalsize
		\item Folgende Funktion wurde als \textbf{veraltet} markiert:\newline
			\fontsize{7.5pt}{10}\selectfont
				\texttt{TYPO3\textbackslash
					CMS\textbackslash
					saltedpasswords\textbackslash
					Utility\textbackslash
					SaltedPasswordsUtility::isUsageEnabled()}
			\normalsize

	\end{itemize}

\end{frame}

% ------------------------------------------------------------------------------
% LTXE-SLIDE-START
% LTXE-SLIDE-UID:		1a5b0bf2-0a03d9a4-25f5eb79-0d3e8d36
% LTXE-SLIDE-TITLE:		Extension lang removed
% LTXE-SLIDE-REFERENCE:	#84680 - Removed unused locallang files from EXT:lang
% LTXE-SLIDE-REFERENCE:	#84680 - Move last language files away from ext:lang and remove ext:lang completely
% ------------------------------------------------------------------------------

\begin{frame}[fragile]
	\frametitle{Veraltete/Entfernte Funktionen}
	\framesubtitle{Die Extension \texttt{EXT:lang} wurde entfernt}

	% decrease font size for code listing
	\lstset{basicstyle=\tiny\ttfamily}

	\begin{itemize}
		\item Nicht verwendete Dateien werden aus der Extension \texttt{EXT:lang} entfernt
		\item Verweise auf die Übersetzungen in \texttt{EXT:lang} geben leere werte zurück
		\item Sprachdateien werden in ihre jeweilige Erweiterung verschoben:
	\end{itemize}

	\begin{lstlisting}
		locallang_alt_intro.xlf => about/Resources/Private/Language/Modules/locallang_alt_intro.xlf
		locallang_alt_doc.xlf => backend/Resources/Private/Language/locallang_alt_doc.xlf
		locallang_login.xlf => backend/Resources/Private/Language/locallang_login.xlf
		locallang_common.xlf => core/Resources/Private/Language/locallang_common.xlf
		locallang_core.xlf => core/Resources/Private/Language/locallang_core.xlf
		locallang_general.xlf => core/Resources/Private/Language/locallang_general.xlf
		locallang_misc.xlf => core/Resources/Private/Language/locallang_misc.xlf
		locallang_mod_web_list.xlf => core/Resources/Private/Language/locallang_mod_web_list.xlf
		locallang_tca.xlf => core/Resources/Private/Language/locallang_tca.xlf
		locallang_tsfe.xlf => core/Resources/Private/Language/locallang_tsfe.xlf
		locallang_wizards.xlf => core/Resources/Private/Language/locallang_wizards.xlf
		locallang_browse_links.xlf => recordlist/Resources/Private/Language/locallang_browse_links.xlf
		locallang_tcemain.xlf => workspaces/Resources/Private/Language/locallang_tcemain.xlf
	\end{lstlisting}

\end{frame}

% ------------------------------------------------------------------------------
% LTXE-SLIDE-START
% LTXE-SLIDE-UID:		1a5b0bf2-0a03d9a4-25f5eb79-0d3e8d36
% LTXE-SLIDE-TITLE:		TSConfig Related Methods
% LTXE-SLIDE-REFERENCE:	#84993 - Deprecate some TSconfig related methods
% ------------------------------------------------------------------------------

\begin{frame}[fragile]
	\frametitle{Veraltete/Entfernte Funktionen}
	\framesubtitle{TSConfig Bezogene Methoden}

	% decrease font size for code listing
	\lstset{basicstyle=\tiny\ttfamily}

	\begin{itemize}
		\item User TSConfig bezogene Methoden wurden als \textbf{veraltet} markiert:

			\begin{lstlisting}
				TYPO3\CMS\core\Authentication\BackendUserAuthentication->getTSConfigVal()
				TYPO3\CMS\core\Authentication\BackendUserAuthentication->getTSConfigProp()
			\end{lstlisting}

		\item Methodensignaturen wurden geändert (Argumente sind nicht mehr erlaubt):

			\begin{lstlisting}
				TYPO3\CMS\core\Authentication\BackendUserAuthentication->getTSConfig()
			\end{lstlisting}

		\item Page TSConfig bezogene Methoden wurden als \textbf{veraltet} markiert:

			\begin{lstlisting}
				TYPO3\CMS\backend\Utility\BackendUtility::getModTSconfig()
				TYPO3\CMS\backend\Utility\BackendUtility::unsetMenuItems()
				TYPO3\CMS\backend\Tree\View\PagePositionMap->getModConfig()
				TYPO3\CMS\core\DataHandling\DataHandler->getTCEMAIN_TSconfig()
			\end{lstlisting}

		\item Eigenschaften die beim Zugriff eine \textbf{deprecation error} Fehlermeldung auslösen:

			\begin{lstlisting}
				TYPO3\CMS\backend\Tree\View\PagePositionMap->getModConfigCache
				TYPO3\CMS\backend\Tree\View\PagePositionMap->modConfigStr
			\end{lstlisting}

	\end{itemize}

\end{frame}


% ------------------------------------------------------------------------------
% LTXE-SLIDE-START
% LTXE-SLIDE-UID:		1a5b0bf2-0a03d9a4-25f5eb79-0d3e8d36
% LTXE-SLIDE-TITLE:		Overriding Page TSConfig
% LTXE-SLIDE-REFERENCE:	#84982 - Overriding page TSconfig mod. with user TSconfig mod.
% ------------------------------------------------------------------------------

\begin{frame}[fragile]
	\frametitle{Veraltete/Entfernte Funktionen}
	\framesubtitle{Überschreiben der Page TSConfig}

	% decrease font size for code listing
	\lstset{basicstyle=\smaller\ttfamily}

	\begin{itemize}
		\item User TSConfig Pfade die mit "\texttt{mod.}" beginnen, lösen einen PHP
			\texttt{E\_USER\_DEPRECATED} Fehler aus und werden in TYPO3 v10 nicht mehr funktionieren
		\item Stellen Sie sicher, dass Sie den User TSConfig Pfad mit "\texttt{page.}" versehen
			wenn ein Page TSConfig Pfad für die Seite auf User TSConfig Ebene überschrieben werden 
			sollte:

			\begin{lstlisting}
				// before
				mod.web_list.disableSingleTableView = 1

				// after
				page.mod.web_list.disableSingleTableView = 1
			\end{lstlisting}

	\end{itemize}

\end{frame}

% ------------------------------------------------------------------------------
% LTXE-SLIDE-START
% LTXE-SLIDE-UID:		1a5b0bf2-0a03d9a4-25f5eb79-0d3e8d36
% LTXE-SLIDE-TITLE:		URL Handlers
% LTXE-SLIDE-REFERENCE:	#85124 - Redirecting urlHandler Hook Concept
% ------------------------------------------------------------------------------

\begin{frame}[fragile]
	\frametitle{Veraltete/Entfernte Funktionen}
	\framesubtitle{URL-Handlers}

	\begin{itemize}
		\item Das  URL-Handler-Konzept, das in TYPO3 v7 eingeführt wurde um die 
			Ausführung von Weiterleitungen zu ermöglichen, wurde als \textbf{veraltet} markiert. 
			Stattdessen sollte PSR-7/PSR-15 Middlewares genutzt werden

		\item Die Ausführung folgender Funktionen löst eine PHP
			\texttt{E\_USER\_DEPRECATED} Fehlermeldung aus:

			\begin{itemize}
				\item \texttt{\$TSFE->initializeRedirectUrlHandlers()}
				\item \texttt{\$TSFE->redirectToExternalUrl()}
			\end{itemize}

	\end{itemize}

\end{frame}

% ------------------------------------------------------------------------------
% LTXE-SLIDE-START
% LTXE-SLIDE-UID:		1a5b0bf2-0a03d9a4-25f5eb79-0d3e8d36
% LTXE-SLIDE-TITLE:		Miscellaneous
% LTXE-SLIDE-REFERENCE:	#81686 - Accessing core TypoScript with .txt file extension has been deprecated
% LTXE-SLIDE-REFERENCE:	#85036 - Removed support for non-namespaced classes in Extbase
% LTXE-SLIDE-REFERENCE:	#85113 - Legacy Backend Module Routing methods
% ------------------------------------------------------------------------------

\begin{frame}[fragile]
	\frametitle{Veraltete/Entfernte Funktionen}
	\framesubtitle{Sonstiges}

	\begin{itemize}
		\item TypoScript Dateien mit der Erweiterung "\texttt{.txt}" wurden in
			"\texttt{.typoscript}" und "\texttt{.tsconfig}" umbenannt
		\item Installationen mit der alten Dateierweiterung werden 
			einen \texttt{E\_USER\_DEPRECATED} PHP Fehler auslösen 
		\item Non-namespaced Klassen wie zum Beispiel
			"\texttt{Tx\_Extension\_Controller\_FooController}"\newline
			sind nicht mehr unterstützt und werden daher nicht mehr funktionieren 
		\item Die folgenden zwei Methoden sind \textbf{veraltet}:

			\begin{itemize}
				\item \texttt{BackendUtility::getModuleUrl()}
				\item \texttt{UriBuilder->buildUriFromModule()}
			\end{itemize}

	\end{itemize}

\end{frame}

% ------------------------------------------------------------------------------
% LTXE-SLIDE-START
% LTXE-SLIDE-UID:		09656e4a-69eb4c46-e38f0664-b3e8a384
% LTXE-SLIDE-TITLE:		More functions...
% ------------------------------------------------------------------------------

\begin{frame}[fragile]
	\frametitle{veraltete/Entfernte Funktionen}

	\vspace{0.6cm}
	\begin{center}
		Viele weitere Funktionen
	\end{center}
	\vspace{-0.8cm}
	\begin{center}
		wurden in der TYPO3 Version 9.2
	\end{center}
	\vspace{-0.8cm}
	\begin{center}
		als veraltet markiert oder entfernt.
	\end{center}
	\vspace{-0.6cm}
	\begin{center}
		Bitte die \href{https://docs.typo3.org/typo3cms/extensions/core/latest/Changelog/9.3/Index.html#deprecation}{TYPO3 Dokumentation} prüfen für weitere Informationen.
	\end{center}

\end{frame}

% ------------------------------------------------------------------------------
