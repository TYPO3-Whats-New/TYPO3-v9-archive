% ------------------------------------------------------------------------------
% TYPO3 Version 9.3 - What's New - Chapter "Changes for Developers" (German Version)
%
% @author	Michael Schams <schams.net>
% @license	Creative Commons BY-NC-SA 3.0
% @link		http://typo3.org/download/release-notes/whats-new/
% @language	German
% ------------------------------------------------------------------------------
% LTXE-CHAPTER-UID:		846d40ce-66fdc2ec-750dcf95-33ce93e0
% LTXE-CHAPTER-NAME:	Changes for Developers
% ------------------------------------------------------------------------------

\section{Änderungen für Entwickler}
\begin{frame}[fragile]
	\frametitle{Änderungen für Entwickler}

	\begin{center}\huge{Kapitel 3:}\end{center}
	\begin{center}\huge{\color{typo3darkgrey}\textbf{Änderungen für Entwickler}}\end{center}

\end{frame}

% ------------------------------------------------------------------------------
% LTXE-SLIDE-START
% LTXE-SLIDE-UID:		f38ee3cb-51d1aa0a-4f6884bf-adadc9e3
% LTXE-SLIDE-TITLE:		Management Database Columns
% LTXE-SLIDE-REFERENCE:	#85160 - Auto create management DB fields from TCA ctrl
% ------------------------------------------------------------------------------

\begin{frame}[fragile]
	\frametitle{Änderungen für Entwickler}
	\framesubtitle{"Management" Datenbankspalten}

	\begin{itemize}
		\item Der Datenbankschema-Analysator erstellt automatisch TYPO3 - "Management"
			Spalten, indem er den TCA liest
		\item Entwickler müssen diese Felder nicht in der Datei 
			\texttt{ext\_tables.sql} angeben
		\item Beispiele für Managementfelder:\newline
			\texttt{uid}, \texttt{pid}, \texttt{crdate}, \texttt{cruser},
			\texttt{hidden}, \texttt{deleted}, \texttt{sortby}, etc.
		\item Felddefinitionen in \texttt{ext\_tables.sql} haben Vorrang vor automatisch 
			generierten Feldern, diese können also bei Bedarf
			angepasst werden
	\end{itemize}

\end{frame}

% ------------------------------------------------------------------------------
% LTXE-SLIDE-START
% LTXE-SLIDE-UID:		f38ee3cb-51d1aa0a-4f6884bf-adadc9e3
% LTXE-SLIDE-TITLE:		Meta Tag Manager (API) (1)
% LTXE-SLIDE-REFERENCE:	#81464 - Add API for meta tag management
% ------------------------------------------------------------------------------

\begin{frame}[fragile]
	\frametitle{Änderungen für Entwickler}
	\framesubtitle{Meta-Tag-Manager (1)}

	% decrease font size for code listing
	\lstset{basicstyle=\tiny\ttfamily}

	\begin{itemize}
		\item Die neue \texttt{MetaTagManager}-API wurde zum Verwalten und Rendern der 
			Metatags auf flexible, geregelte Weise eingeführt.
		\item TYPO3 core liefert zum Beispiel einen \href{http://ogp.me/}{Open Graph}
			\texttt{MetaTagManager}

			\begin{lstlisting}
				use \TYPO3\CMS\Core\MetaTag\MetaTagManagerRegistry;
				$metaTagManager = MetaTagManagerRegistry::getInstance()->getManagerForProperty('og:title');
				$metaTagManager->addProperty('og:title', 'This is the OG title from a controller');
			\end{lstlisting}

		\item Weitere verfügbare Beispielfunktionen:

			\begin{itemize}
				\smaller
					\item \texttt{\$metaTagManager->addProperty()}
					\item \texttt{\$metaTagManager->removeProperty()}
					\item \texttt{\$metaTagManager->removeAllProperties()}
			\end{itemize}

	\end{itemize}

\end{frame}

% ------------------------------------------------------------------------------
% LTXE-SLIDE-START
% LTXE-SLIDE-UID:		f38ee3cb-51d1aa0a-4f6884bf-adadc9e3
% LTXE-SLIDE-TITLE:		Meta Tag Manager (API) (2)
% LTXE-SLIDE-REFERENCE:	#81464 - Add API for meta tag management
% ------------------------------------------------------------------------------

\begin{frame}[fragile]
	\frametitle{Änderungen für Entwickler}
	\framesubtitle{Meta-Tag-Manager (2)}

	% decrease font size for code listing
	\lstset{basicstyle=\tiny\ttfamily}

	\begin{itemize}
		\item Die Entwickler können benutzerdefinierte \texttt{MetaTagManager} in der
			\texttt{MetaTagManagerRegistry} anlegen

			\begin{lstlisting}
				use \TYPO3\CMS\Core\MetaTag\MetaTagManagerRegistry;
				$metaTagManagerRegistry = MetaTagManagerRegistry::getInstance();
				$metaTagManagerRegistry->registerManager(
				  'custom',
				  \Some\CustomExtension\MetaTag\CustomMetaTagManager::class
				);
			\end{lstlisting}

		\item Meta-Tags können mit TypoScript und PHP gesetzt werden

			\begin{lstlisting}
				page.meta {
				  og:site_name = TYPO3
				  og:site_name.attribute = property
				  og:site_name.replace = 1
				}
			\end{lstlisting}

			\smaller
				("\texttt{replace = 1}" ersetzt zuvor festgelegte Meta-Tags)
			\normalsize

	\end{itemize}

\end{frame}

% ------------------------------------------------------------------------------
% LTXE-SLIDE-START
% LTXE-SLIDE-UID:		f38ee3cb-51d1aa0a-4f6884bf-adadc9e3
% LTXE-SLIDE-TITLE:		Doctrine: Negative DateInterval Fields
% LTXE-SLIDE-REFERENCE:	#84744 - Raise doctrine/dbal-version
% ------------------------------------------------------------------------------

\begin{frame}[fragile]
	\frametitle{Änderungen für Entwickler}
	\framesubtitle{Doctrine: Negative DateInterval Felder}

	\begin{itemize}
		\item "Doctrine" wurde auf Version 2.7.1 erhöht
		\item Der Werte der \texttt{DateInterval}-Felder können nun auch negativ sein, 
			das heißt, dass sie entweder mit "\texttt{+}" oder "\texttt{-}" beginnen müssen
		\item Migration: angenommen, dass negative \texttt{DateIntervals} noch nicht verwendet worden sind,
			einfach die Daten mit "\texttt{+}" voranstellen
	\end{itemize}

	\breakingchange

\end{frame}

% ------------------------------------------------------------------------------
% LTXE-SLIDE-START
% LTXE-SLIDE-UID:		f38ee3cb-51d1aa0a-4f6884bf-adadc9e3
% LTXE-SLIDE-TITLE:		Validate annotation
% LTXE-SLIDE-REFERENCE:	#83167 - Replace @validate with @TYPO3\CMS\Extbase\Annotation\Validate
% ------------------------------------------------------------------------------

\begin{frame}[fragile]
	\frametitle{Änderungen für Entwickler }
	\framesubtitle{Annotation als Klassennamen validieren}

	% decrease font size for code listing
	\lstset{basicstyle=\tiny\ttfamily}

	\begin{itemize}
		\item Die Doctrine Annotation
			\texttt{TYPO3\textbackslash
				CMS\textbackslash
				Extbase\textbackslash
				Annotation\textbackslash
				Validate} wurde eingeführt
		\item Dies ist ein Nachfolger der Annotation \texttt{\@validate} 
		\item Ein Beispiel dafür:

			\begin{lstlisting}
				/**
				 * @TYPO3\CMS\Extbase\Annotation\Validate
				 * @var Foo
				 */
				public $property;
			\end{lstlisting}

		\item Die \texttt{use}-Anweisung kann ebenfalls verwendet werden:

			\begin{lstlisting}
				use TYPO3\CMS\Extbase\Annotation\Validate;

				/**
				 * @Validate
				 * @var Foo
				 */
				public $property;
			\end{lstlisting}

	\end{itemize}

\end{frame}

% ------------------------------------------------------------------------------
% LTXE-SLIDE-START
% LTXE-SLIDE-UID:		f38ee3cb-51d1aa0a-4f6884bf-adadc9e3
% LTXE-SLIDE-TITLE:		Backend ViewHelpers
% LTXE-SLIDE-REFERENCE:	#83983 - Improved ModuleLinkViewHelper
% LTXE-SLIDE-REFERENCE:	#84983 - BE ViewHelper for EditDocumentController
% ------------------------------------------------------------------------------

\begin{frame}[fragile]
	\frametitle{Änderungen für Entwickler}
	\framesubtitle{Backend ViewHelpers}

	% decrease font size for code listing
	\lstset{basicstyle=\fontsize{7.5pt}{10}\selectfont\ttfamily}

	\begin{itemize}
		\item Das Modul Link ViewHelper unterstützt zwei neue Argumente\newline
			\small
				\texttt{TYPO3\textbackslash
					CMS\textbackslash
					Backend\textbackslash
					ViewHelpers\textbackslash
					ModuleLinkViewHelper}
			\normalsize

			\begin{itemize}
				\small
				\item \texttt{query}: erlaubt Abfrageparameter auch als String zu definieren
				\item \texttt{currentUrlParameterName}: Argument verwendet die aktuelle URL
			\end{itemize}

			Diese Änderung ermöglicht es Entwicklern, vorhandene benutzerdefinierte Viewhelper Backend-Routen
			zu diesem Viewhelper zu migrieren.

		\item Neue ViewHelper-Funktionen für das Backend zur Vereinfachung der Erstellung/Bearbeitung der Datensätze:

			\begin{lstlisting}
				<be:uri.newRecord pid=" ... " table=" ... " />
				<be:link.newRecord pid=" ... " table=" ... " />
				<be:uri.editRecord uid=" ... " table=" ... " />
				<be:link.editRecord uid=" ... " table=" ... " />
			\end{lstlisting}

	\end{itemize}

\end{frame}

% ------------------------------------------------------------------------------
% LTXE-SLIDE-START
% LTXE-SLIDE-UID:		f38ee3cb-51d1aa0a-4f6884bf-adadc9e3
% LTXE-SLIDE-TITLE:		LanguageMenu Processor (1)
% LTXE-SLIDE-REFERENCE:	#84650 - Introduce fluid data processor for language menus
% ------------------------------------------------------------------------------

\begin{frame}[fragile]
	\frametitle{Änderungen für Entwickler}
	\framesubtitle{LanguageMenu Processor (1)}

	% decrease font size for code listing
	\lstset{basicstyle=\tiny\ttfamily}

	\begin{itemize}
		\item Ein neuer \texttt{LanguageMenuProcessor} für Fluid wurde eingeführt

			\begin{lstlisting}
				10 = TYPO3\CMS\Frontend\DataProcessing\LanguageMenuProcessor
				10 {
				  languages = auto
				  as = languageNavigation
				}
			\end{lstlisting}

		\item mögliche Optionen:

			\begin{itemize}

				\item \texttt{if}: TypoScript \texttt{if} Bedingung
				\item \texttt{languages}: eine Liste von durch Komma getrennten Sprach-IDs
					(z.B. 0,1,2) "\texttt{auto}" zum laden von Seitensprachen
				\item \texttt{as}: Variable, die innerhalb des Ergebnisses verwendet werden soll
			\end{itemize}

	\end{itemize}

\end{frame}

% ------------------------------------------------------------------------------
% LTXE-SLIDE-START
% LTXE-SLIDE-UID:		f38ee3cb-51d1aa0a-4f6884bf-adadc9e3
% LTXE-SLIDE-TITLE:		LanguageMenu Processor (2)
% LTXE-SLIDE-REFERENCE:	#84650 - Introduce fluid data processor for language menus
% ------------------------------------------------------------------------------

\begin{frame}[fragile]
	\frametitle{Änderungen für Entwickler}
	\framesubtitle{LanguageMenu Processor (2)}

	% decrease font size for code listing
	\lstset{basicstyle=\tiny\ttfamily}

	\begin{itemize}
		\item Ein Beispiel für Fluid-Vorlage:

			\begin{lstlisting}
				<f:if condition="{languageNavigation}">
				  <ul id="language" class="language-menu">
				    <f:for each="{languageNavigation}" as="item">
				      <li class="{f:if(condition: item.active, then: 'active')}{f:if(condition: item.available, else: ' text-muted')}">
				        <f:if condition="{item.available}">
				          <f:then>
				            <a href="{item.link}" hreflang="{item.hreflang}" title="{item.navigationTitle}">
				              <span>{item.navigationTitle}</span>
				            </a>
				          </f:then>
				          <f:else>
				            <span>{item.navigationTitle}</span>
				          </f:else>
				        </f:if>
				      </li>
				    </f:for>
				  </ul>
				</f:if>
			\end{lstlisting}

	\end{itemize}

\end{frame}

% ------------------------------------------------------------------------------
% LTXE-SLIDE-START
% LTXE-SLIDE-UID:		f38ee3cb-51d1aa0a-4f6884bf-adadc9e3
% LTXE-SLIDE-TITLE:		Miscellaneous (1)
% LTXE-SLIDE-REFERENCE:	#85025 - Enumerations are now final
% LTXE-SLIDE-REFERENCE:	#84244 - Allow adding additional query restrictions
% ------------------------------------------------------------------------------

\begin{frame}[fragile]
	\frametitle{Änderungen für Entwickler}
	\framesubtitle{Sonstiges}

	\begin{itemize}
		\item Alle TYPO3 ennumeration-Klassen wurden als "\texttt{final}" markiert und Klassen
			 von Drittanbietern, die Aufzählungen erweitern, lösen einen schwerwiegenden PHP-Fehler aus
		\item Zusätzliche Abfrageeinschränkungen können zur \newline
			\smaller
				\texttt{\$GLOBALS['TYPO3\_CONF\_VARS']['DB']['additionalQueryRestrictions']} 
			\normalsize hinzugefügt werden \newline
			Diese Begrenzungen sollten zu jeder ausgewählten Abfrage hinzugefügt werden, die 
			mit dem QueryBuilder ausgeführt wird (vorsichtig verwenden!)

	\end{itemize}

\end{frame}

% ------------------------------------------------------------------------------
