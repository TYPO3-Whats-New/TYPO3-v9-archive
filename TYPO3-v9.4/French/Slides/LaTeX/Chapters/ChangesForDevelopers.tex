% ------------------------------------------------------------------------------
% TYPO3 Version 9.4 - What's New (French Version)
%
% @license	Creative Commons BY-NC-SA 3.0
% @link		https://typo3.org/help/documentation/whats-new/
% @language	French
% ------------------------------------------------------------------------------

\section{Changements pour les développeurs}
\begin{frame}[fragile]
	\frametitle{Changements pour les développeurs}

	\begin{center}\huge{Chapitre 3~:}\end{center}
	\begin{center}\huge{\color{typo3darkgrey}\textbf{Changements pour les développeurs}}\end{center}

\end{frame}

% ------------------------------------------------------------------------------
% #85389 - Introduce Context API for consistent data handling

\begin{frame}[fragile]
	\frametitle{Changements pour les développeurs}
	\framesubtitle{Context API}

	\begin{itemize}
		\item La Context API est introduite dans TYPO3 version 9.4
		\item L'objectif principal est de centraliser des variables globales
		\item Son but est de remplacer les objets accessibles globalement (comme
			\texttt{TSFE}, \texttt{sys\_page}, \texttt{BE\_USER}, etc.) et de
			les rendrent disponibles d'une manière commune, structurée et logique
		\item Au lieu d'exposer un objet complet (i.e. l'objet \texttt{BE\_USER}),
			un «~aspect~» contient seulement les propriétés qui sont pertinantes
			et requises
		\item Les développeurs d'extension peuvent ajouter des aspects au contexte actuel
		\item Consulter \href{https://docs.typo3.org/typo3cms/extensions/core/latest/Changelog/9.4/Feature-85389-ContextAPIForConsistentDataHandling.html}{documentation}
			pour plus d'information et des exemples d'utilisation de l'API
	\end{itemize}

\end{frame}

% ------------------------------------------------------------------------------
% #85590 - Hooks for DatabaseRecordList CSV actions

\begin{frame}[fragile]
	\frametitle{Changements pour les développeurs}
	\framesubtitle{Personaliser les fichiers CSV}

	% decrease font size for code listing
	\lstset{basicstyle=\tiny\ttfamily}

	\begin{itemize}
		\item Lors de l'export d'enregistrements en CSV, il est possible de manipuler
			la sortie avant le début du téléchargement
		\item Les nouveaux hooks suivants permettent de réaliser l'opération~:

			\begin{itemize}
			\smaller
				\item \texttt{customizeCsvHeader} - pour personaliser l'en-tête
				\item \texttt{customizeCsvRow} - pour personaliser une ligne individuellement
			\end{itemize}

		\item Exemple d'utilisation~:

			\begin{lstlisting}
use \TYPO3\CMS\Recordlist\RecordList\DatabaseRecordList;

$hookName = DatabaseRecordList::class;
$GLOBALS['TYPO3_CONF_VARS']['SC_OPTIONS'][$hookName]['customizeCsvRow'][] =
  \Vendor\ExtName\Hooks\CsvTest::class . '->customizeCsvRow';
$GLOBALS['TYPO3_CONF_VARS']['SC_OPTIONS'][$hookName]['customizeCsvHeader'][] =
  \Vendor\ExtName\Hooks\CsvTest::class . '->customizeCsvHeader';
\end{lstlisting}

	\end{itemize}

\end{frame}

% ------------------------------------------------------------------------------
% #85828 - Move symfony expression language handling into EXT:core

\begin{frame}[fragile]
	\frametitle{Changements pour les développeurs}
	\framesubtitle{Language d'expression Symfony}

	% decrease font size for code listing
	\lstset{basicstyle=\tiny\ttfamily}

	\begin{itemize}
		\item Le \href{https://symfony.com/doc/current/components/expression_language/syntax.html}{Symfony Expression Language}
			est déplacé de \texttt{EXT:form} dans le cœur de TYPO3
		\item Par ce déplacement, ce language d'expression devient disponible ailleurs
		\item La classe de fonctionnalité du cœur \texttt{DefaultProvider}, peut s'utiliser
			directement (voir exemple ci-dessous) et les implémentations personalisées
			peuvent étendre la classe \texttt{AbstractProvider}

			\begin{lstlisting}
use \TYPO3\CMS\Core\ExpressionLanguage\DefaultProvider;
use \TYPO3\CMS\Core\ExpressionLanguage\Resolver;

$provider = GeneralUtility::makeInstance(DefaultProvider::class);
$conditionResolver = GeneralUtility::makeInstance(Resolver::class, $provider);
$conditionResolver->evaluate('1 < 2'); // result is true
			\end{lstlisting}

	\end{itemize}

\end{frame}

% ------------------------------------------------------------------------------
% #85356 - Add support to CurrencyViewHelper for mdash

\begin{frame}[fragile]
	\frametitle{Changements pour les développeurs}
	\framesubtitle{ViewHelper pour les devises}

%	% decrease font size for code listing
%	\lstset{basicstyle=\tiny\ttfamily}

	\begin{itemize}
		\item Il est possible d'utiliser un tiret au lieu des décimales \texttt{00}
			dans le ViewHelper Currency
		\item L'option \texttt{useDash="1"} active cette fonction
		\item L'exemple suivant retourne \texttt{123.-}

			\begin{lstlisting}
<f:format.currency useDash="1">123.00</f:format.currency>
			\end{lstlisting}

	\end{itemize}

\end{frame}

% ------------------------------------------------------------------------------
% #85699 - Deprecate methods in PageRepository
% #85727 - Move DatabaseIntegrityCheck to EXT:lowlevel

\begin{frame}[fragile]
	\frametitle{Changements pour les développeurs}
	\framesubtitle{Changements dans le classe \texttt{PageRepository}}

	\begin{itemize}
		\item Les méthodes suivantes sont marquées \textit{internal}~:

			\begin{itemize}
			\smaller
				\item \texttt{TYPO3\textbackslash
					CMS\textbackslash
					Frontend\textbackslash
					Page\textbackslash
					PageRepository::getMovePlaceholder()}

				\item \texttt{TYPO3\textbackslash
					CMS\textbackslash
					Frontend\textbackslash
					Page\textbackslash
					PageRepository::movePlhOL()}
			\end{itemize}

		\item La classe suivante est déplacée du cœur de TYPO3 vers l'extension
			système \texttt{lowlevel}~:

			\begin{itemize}
			\smaller
				\item \texttt{TYPO3\textbackslash
					CMS\textbackslash
					Core\textbackslash
					Integrity\textbackslash
					DatabaseIntegrityCheck}
			\end{itemize}

	\end{itemize}

\end{frame}

% ------------------------------------------------------------------------------
% #85801 - Deprecate 2nd argument of GeneralUtility::explodeUrl2Array

\begin{frame}[fragile]
	\frametitle{Changements pour les développeurs}
	\framesubtitle{Méthode \texttt{GeneralUtility::explodeUrl2Array()}}

	\begin{itemize}
		\item Le second argument de la méthode
			\smaller
				\texttt{TYPO3\textbackslash
					CMS\textbackslash
					Core\textbackslash
					Utility\textbackslash
					GeneralUtility::explodeUrl2Array()}\newline
			\normalsize
			ne doit plus être utilisé (marqué déprécié)

		\item Si défini, la chaîne passée en premier argument était transformée
			en tableau multidimensionel lorsque les crochets étaient utilisés dans les
			noms de variables
		\item Il faut simplifier en utilisant la méthode native de PHP \texttt{parse\_str()}~:

			\begin{lstlisting}
$result = [];
parse_str($queryParametersAsString, $result);
			\end{lstlisting}

	\end{itemize}

\end{frame}

% ------------------------------------------------------------------------------
% #84280 - Make unit tests notice free

\begin{frame}[fragile]
	\frametitle{Changements pour les développeurs}
	\framesubtitle{Tests unitaires}

	\begin{itemize}
		\item Les tests unitaires du cœur de TYPO3 ne provoquent plus d'\texttt{E\_NOTICE}
		\item En conséquence, l'option \texttt{\$suppressNotices = true} est retirée et la version
			du framework de test passée à 4.4.0

	\end{itemize}

\end{frame}

% ------------------------------------------------------------------------------
