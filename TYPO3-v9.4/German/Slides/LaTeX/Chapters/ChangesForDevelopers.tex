% ------------------------------------------------------------------------------
% TYPO3 Version 9.4 - What's New (German Version)
%
% @license	Creative Commons BY-NC-SA 3.0
% @link		https://typo3.org/help/documentation/whats-new/
% @language	German
% ------------------------------------------------------------------------------

\section{Änderungen für Entwickler}
\begin{frame}[fragile]
	\frametitle{Änderungen für Entwickler}

	\begin{center}\huge{Kapitel 3:}\end{center}
	\begin{center}\huge{\color{typo3darkgrey}\textbf{Änderungen für Entwickler}}\end{center}

\end{frame}

% ------------------------------------------------------------------------------
% #85389 - Introduce Context API for consistent data handling

\begin{frame}[fragile]
	\frametitle{Änderungen für Entwickler}
	\framesubtitle{Kontext-API}

	\begin{itemize}
		\item Eine neue Kontext-API wurde in der TYPO3 Version 9.4 eingeführt
		\item Hauptziel dieses Konzepts ist globale Variablen zu zentralisieren
		\item Der API zielt darauf ab, global verfügbare Objekte (z.B.
			\texttt{TSFE}, \texttt{sys\_page}, \texttt{BE\_USER}, usw.) zu ersezten
			und sie in einer logischen Weise zu strukturieren und verfügbar zu machen
		\item Anstatt ein vollständiges Objekt (z.B. das Objekt the \texttt{BE\_USER})
			 anzuzeigen , die "aspects" enthalten Eigenschaften, die relevant
			und erforderlich sind
		\item Erweiterungsentwickler können Aspekte zum aktuellen Kontext hinzufügen
		\item Für weitere Details und Beispiele zur Verwendung der API bitte die \href{https://docs.typo3.org/typo3cms/extensions/core/latest/Changelog/9.4/Feature-85389-ContextAPIForConsistentDataHandling.rst}{Dokumentation}
			prüfen
	\end{itemize}

\end{frame}

% ------------------------------------------------------------------------------
% #85590 - Hooks for DatabaseRecordList CSV actions

\begin{frame}[fragile]
	\frametitle{Änderungen für Entwickler}
	\framesubtitle{CSV-Dateien individualisieren}

	% decrease font size for code listing
	\lstset{basicstyle=\tiny\ttfamily}

	\begin{itemize}
		\item Die Ausgabe beim Exportieren von Datenbankeinträgen als CSV kann
			vor dem Start des Downloads manipuliert werden.
		\item Dies wird durch folgenden zwei Hooks ermöglicht:

			\begin{itemize}
			\smaller
				\item \texttt{customizeCsvHeader} - um das Header anzupassen
				\item \texttt{customizeCsvRow} - um eine einzelne Zeile anzupassen
			\end{itemize}

		\item Verwendungsbeispiele:

			\begin{lstlisting}
use \TYPO3\CMS\Recordlist\RecordList\DatabaseRecordList;

$hookName = DatabaseRecordList::class;
$GLOBALS['TYPO3_CONF_VARS']['SC_OPTIONS'][$hookName]['customizeCsvRow'][] =
  \Vendor\ExtName\Hooks\CsvTest::class . '->customizeCsvRow';
$GLOBALS['TYPO3_CONF_VARS']['SC_OPTIONS'][$hookName]['customizeCsvHeader'][] =
  \Vendor\ExtName\Hooks\CsvTest::class . '->customizeCsvHeader';
\end{lstlisting}

	\end{itemize}

\end{frame}

% ------------------------------------------------------------------------------
% #85828 - Move symfony expression language handling into EXT:core

\begin{frame}[fragile]
	\frametitle{Änderungen für Entwickler}
	\framesubtitle{Symfony Ausdruckssprache}

	% decrease font size for code listing
	\lstset{basicstyle=\tiny\ttfamily}

	\begin{itemize}
		\item Die \href{https://symfony.com/doc/current/components/expression_language/syntax.html}{Symfony Ausdruckssprache}
			 wurde von \texttt{EXT:form} in den TYPO3-Core verschoben
		\item Durch die Verschiebung in den Core, ist nun diese Ausdruckssprache 
			auch in anderen Bereichen verfügbar
		\item Der TYPO3-Core enthält die Klasse \texttt{DefaultProvider}, die direkt verwendet werden
			kann (siehe Beispiel unten). Benutzerdefinierte Implementierungen können
			die Klasse \texttt{AbstractProvider} erweitern

			\begin{lstlisting}
use \TYPO3\CMS\Core\ExpressionLanguage\DefaultProvider;
use \TYPO3\CMS\Core\ExpressionLanguage\Resolver;

$provider = GeneralUtility::makeInstance(DefaultProvider::class);
$conditionResolver = GeneralUtility::makeInstance(Resolver::class, $provider);
$conditionResolver->evaluate('1 < 2'); // result is true
			\end{lstlisting}

	\end{itemize}

\end{frame}

% ------------------------------------------------------------------------------
% #85356 - Add support to CurrencyViewHelper for mdash

\begin{frame}[fragile]
	\frametitle{Änderungen für Entwickler}
	\framesubtitle{"Currency" ViewHelper}

%	% decrease font size for code listing
%	\lstset{basicstyle=\tiny\ttfamily}

	\begin{itemize}
		\item Ein Bindestrich kann anstelle von \texttt{00} im 
			Currency ViewHelper verwendet werden
		\item Die Option \texttt{useDash="1"} ermöglicht diese Funktion
		\item Folgendes Beispiel stellt das Ergebnis \texttt{123.-} dar

			\begin{lstlisting}
<f:format.currency useDash="1">123.00</f:format.currency>
			\end{lstlisting}

	\end{itemize}

\end{frame}

% ------------------------------------------------------------------------------
% #85699 - Deprecate methods in PageRepository
% #85727 - Move DatabaseIntegrityCheck to EXT:lowlevel

\begin{frame}[fragile]
	\frametitle{Änderungen für Entwickler}
	\framesubtitle{Änderungen in der \texttt{PageRepository} Klasse}

	\begin{itemize}
		\item Folgende Methoden wurden als \textit{internal} markiert:

			\begin{itemize}
			\smaller
				\item \texttt{TYPO3\textbackslash
					CMS\textbackslash
					Frontend\textbackslash
					Page\textbackslash
					PageRepository::getMovePlaceholder()}

				\item \texttt{TYPO3\textbackslash
					CMS\textbackslash
					Frontend\textbackslash
					Page\textbackslash
					PageRepository::movePlhOL()}
			\end{itemize}

		\item Folgende Klasse wurde aus dem TYPO3-Core in den
			Systemerweiterung \texttt{lowlevel} verschoben:

			\begin{itemize}
			\smaller
				\item \texttt{TYPO3\textbackslash
					CMS\textbackslash
					Core\textbackslash
					Integrity\textbackslash
					DatabaseIntegrityCheck}
			\end{itemize}

	\end{itemize}

\end{frame}

% ------------------------------------------------------------------------------
% #85801 - Deprecate 2nd argument of GeneralUtility::explodeUrl2Array

\begin{frame}[fragile]
	\frametitle{Änderungen für Entwickler}
	\framesubtitle{Die Methode \texttt{GeneralUtility::explodeUrl2Array()}}

	\begin{itemize}
		\item Zweites Argument der Methode
			\smaller
				\texttt{TYPO3\textbackslash
					CMS\textbackslash
					Core\textbackslash
					Utility\textbackslash
					GeneralUtility::explodeUrl2Array()}\newline
			\normalsize
			wurde als veraltet markiert und sollte nicht mehr verwendet werden

		\item Wenn gesetzt, der String des ersten Arguments wurde in ein mehrdimensionales Arrey geparst
			falls in Variablennamen eckige Klammern verwendet werden
		\item Dies kann vereinfacht werden mit Hilfe der \texttt{parse\_str()}
			Methode:

			\begin{lstlisting}
$result = [];
parse_str($queryParametersAsString, $result);
			\end{lstlisting}

	\end{itemize}

\end{frame}

% ------------------------------------------------------------------------------
% #84280 - Make unit tests notice free

\begin{frame}[fragile]
	\frametitle{Änderungen für Entwickler}
	\framesubtitle{Unit Tests}

	\begin{itemize}
		\item TYPO3-Core Unit Tests lösen nicht mehr eine \texttt{E\_NOTICE} Fehlermeldung aus
		\item Als Konsequenz wurde die Flagge \texttt{\$suppressNotices = true} entfernt und 
			das Testrahmenwerk wurde auf Version 4.4.0 erhöht

	\end{itemize}

\end{frame}

% ------------------------------------------------------------------------------
