% ------------------------------------------------------------------------------
% TYPO3 Version 9.4 - What's New (English Version)
%
% @author	Michael Schams <schams.net>
% @license	Creative Commons BY-NC-SA 3.0
% @link		https://typo3.org/help/documentation/whats-new/
% @language	English
% ------------------------------------------------------------------------------

\section{Changes for Developers}
\begin{frame}[fragile]
	\frametitle{Changes for Developers}

	\begin{center}\huge{Chapter 3:}\end{center}
	\begin{center}\huge{\color{typo3darkgrey}\textbf{Changes for Developers}}\end{center}

\end{frame}

% ------------------------------------------------------------------------------
% #85389 - Introduce Context API for consistent data handling

\begin{frame}[fragile]
	\frametitle{Changes for Developers}
	\framesubtitle{Context API}

	\begin{itemize}
		\item A new Context API has been introduced in TYPO3 version 9.4
		\item The main goal of this concept is to centralize global variables
		\item The API aims to replace globally available objects (e.g.
			\texttt{TSFE}, \texttt{sys\_page}, \texttt{BE\_USER}, etc.) and to
			make them available in a common, structured and logical way
		\item Instead of exposing a full object (e.g. the \texttt{BE\_USER}
			object), "aspects" contain properties, which are relevant and
			required only
		\item Extension developers can add aspects to the current context
		\item See \href{https://docs.typo3.org/typo3cms/extensions/core/latest/Changelog/9.4/Feature-85389-ContextAPIForConsistentDataHandling.html}{documentation}
			for further details and examples how to use the API
	\end{itemize}

\end{frame}

% ------------------------------------------------------------------------------
% #85590 - Hooks for DatabaseRecordList CSV actions

\begin{frame}[fragile]
	\frametitle{Changes for Developers}
	\framesubtitle{Customizing CSV Files}

	% decrease font size for code listing
	\lstset{basicstyle=\tiny\ttfamily}

	\begin{itemize}
		\item When exporting database records as CSV, the output can be
			manipulated before the download starts
		\item The following two new hooks allow developers to achieve this:

			\begin{itemize}
			\smaller
				\item \texttt{customizeCsvHeader} - to customize the header
				\item \texttt{customizeCsvRow} - to customize a single row
			\end{itemize}

		\item Usage example:

			\begin{lstlisting}
use \TYPO3\CMS\Recordlist\RecordList\DatabaseRecordList;

$hookName = DatabaseRecordList::class;
$GLOBALS['TYPO3_CONF_VARS']['SC_OPTIONS'][$hookName]['customizeCsvRow'][] =
  \Vendor\ExtName\Hooks\CsvTest::class . '->customizeCsvRow';
$GLOBALS['TYPO3_CONF_VARS']['SC_OPTIONS'][$hookName]['customizeCsvHeader'][] =
  \Vendor\ExtName\Hooks\CsvTest::class . '->customizeCsvHeader';
\end{lstlisting}

	\end{itemize}

\end{frame}

% ------------------------------------------------------------------------------
% #85828 - Move symfony expression language handling into EXT:core

\begin{frame}[fragile]
	\frametitle{Changes for Developers}
	\framesubtitle{Symfony Expression Language}

	% decrease font size for code listing
	\lstset{basicstyle=\tiny\ttfamily}

	\begin{itemize}
		\item The \href{https://symfony.com/doc/current/components/expression_language/syntax.html}{Symfony Expression Language}
			has been moved from \texttt{EXT:form} into the TYPO3 core
		\item By moving it into the core, the expression language is now also
			available in other scopes
		\item The TYPO3 core features class \texttt{DefaultProvider}, which can be used
			directly (see example below) and custom implementations can extend
			class \texttt{AbstractProvider}

			\begin{lstlisting}
use \TYPO3\CMS\Core\ExpressionLanguage\DefaultProvider;
use \TYPO3\CMS\Core\ExpressionLanguage\Resolver;

$provider = GeneralUtility::makeInstance(DefaultProvider::class);
$conditionResolver = GeneralUtility::makeInstance(Resolver::class, $provider);
$conditionResolver->evaluate('1 < 2'); // result is true
			\end{lstlisting}

	\end{itemize}

\end{frame}

% ------------------------------------------------------------------------------
% #85356 - Add support to CurrencyViewHelper for mdash

\begin{frame}[fragile]
	\frametitle{Changes for Developers}
	\framesubtitle{"Currency" ViewHelper}

%	% decrease font size for code listing
%	\lstset{basicstyle=\tiny\ttfamily}

	\begin{itemize}
		\item A dash can be used instead of decimal \texttt{00} in the
			Currency ViewHelper
		\item Option \texttt{useDash="1"} enables this feature
		\item The following example outputs \texttt{123.-}

			\begin{lstlisting}
<f:format.currency useDash="1">123.00</f:format.currency>
			\end{lstlisting}

	\end{itemize}

\end{frame}

% ------------------------------------------------------------------------------
% #85699 - Deprecate methods in PageRepository
% #85727 - Move DatabaseIntegrityCheck to EXT:lowlevel

\begin{frame}[fragile]
	\frametitle{Changes for Developers}
	\framesubtitle{Changes in \texttt{PageRepository} Class}

	\begin{itemize}
		\item The following methods have been marked \textit{internal}:

			\begin{itemize}
			\smaller
				\item \texttt{TYPO3\textbackslash
					CMS\textbackslash
					Frontend\textbackslash
					Page\textbackslash
					PageRepository::getMovePlaceholder()}

				\item \texttt{TYPO3\textbackslash
					CMS\textbackslash
					Frontend\textbackslash
					Page\textbackslash
					PageRepository::movePlhOL()}
			\end{itemize}

		\item The following class has been moved from the TYPO3 core to the
			system extension \texttt{lowlevel}:

			\begin{itemize}
			\smaller
				\item \texttt{TYPO3\textbackslash
					CMS\textbackslash
					Core\textbackslash
					Integrity\textbackslash
					DatabaseIntegrityCheck}
			\end{itemize}

	\end{itemize}

\end{frame}

% ------------------------------------------------------------------------------
% #85801 - Deprecate 2nd argument of GeneralUtility::explodeUrl2Array

\begin{frame}[fragile]
	\frametitle{Changes for Developers}
	\framesubtitle{Method \texttt{GeneralUtility::explodeUrl2Array()}}

	\begin{itemize}
		\item Second argument of method
			\smaller
				\texttt{TYPO3\textbackslash
					CMS\textbackslash
					Core\textbackslash
					Utility\textbackslash
					GeneralUtility::explodeUrl2Array()}\newline
			\normalsize
			should not be used anymore (marked deprecated)

		\item If set, the string of the 1st argument was parsed into a
			multi-dimensional array if square brackets are used in variable names
		\item This can be simplified by using PHP's native \texttt{parse\_str()}
			method:

			\begin{lstlisting}
$result = [];
parse_str($queryParametersAsString, $result);
			\end{lstlisting}

	\end{itemize}

\end{frame}

% ------------------------------------------------------------------------------
% #84280 - Make unit tests notice free

\begin{frame}[fragile]
	\frametitle{Changes for Developers}
	\framesubtitle{Unit Tests}

	\begin{itemize}
		\item TYPO3 core unit tests do not trigger an \texttt{E\_NOTICE} anymore
		\item As a consequence, flag \texttt{\$suppressNotices = true} has been
			removed and the testing-framework raised to version 4.4.0

	\end{itemize}

\end{frame}

% ------------------------------------------------------------------------------
