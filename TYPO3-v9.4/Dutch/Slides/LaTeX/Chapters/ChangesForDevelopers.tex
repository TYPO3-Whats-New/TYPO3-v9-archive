% ------------------------------------------------------------------------------
% TYPO3 Version 9.4 - What's New (Dutch Version)
%
% @license	Creative Commons BY-NC-SA 3.0
% @link		https://typo3.org/help/documentation/whats-new/
% @language	Dutch
% ------------------------------------------------------------------------------

\section{Wijzigingen voor ontwikkelaars}
\begin{frame}[fragile]
	\frametitle{Wijzigingen voor ontwikkelaars}

	\begin{center}\huge{Hoofdstuk 3:}\end{center}
	\begin{center}\huge{\color{typo3darkgrey}\textbf{Wijzigingen voor ontwikkelaars}}\end{center}

\end{frame}

% ------------------------------------------------------------------------------
% #85389 - Introduce Context API for consistent data handling

\begin{frame}[fragile]
	\frametitle{Wijzigingen voor ontwikkelaars}
	\framesubtitle{Context API}

	\begin{itemize}
		\item Een nieuwe Context API is geïntroduceerd in TYPO3 versie 9.4
		\item Het hoofddoel van dit concept is het centraliseren van globale variabelen
		\item De API heeft als doel het vervangen van globale objecten (bijv.
			\texttt{TSFE}, \texttt{sys\_page}, \texttt{BE\_USER}, etc.) en deze
			beschikbaar te stellen op een gestructureerde en logische manier.
		\item In plaats van een heel object (bijv. \texttt{BE\_USER}) bevatten
			"aspecten" eigenschappen die relevant en alleen vereist zijn
		\item Extensieontwikkelaars kunnen aspeccten toevoegen aan de huidige
			context
		\item Zie \href{https://docs.typo3.org/typo3cms/extensions/core/latest/Changelog/9.4/Feature-85389-ContextAPIForConsistentDataHandling.rst}{documentatie}
			voor meer details en voorbeelden voor gebruik van de API
	\end{itemize}

\end{frame}

% ------------------------------------------------------------------------------
% #85590 - Hooks for DatabaseRecordList CSV actions

\begin{frame}[fragile]
	\frametitle{Wijzigingen voor ontwikkelaars}
	\framesubtitle{CSV-bestanden aanpassen}

	% decrease font size for code listing
	\lstset{basicstyle=\tiny\ttfamily}

	\begin{itemize}
		\item BIj het exporteren van records als CSV kan de uitvoer
			aangepast worden voordat de download begint
		\item Met de volgende twee nieuwe hooks kan dit bereikt worden:

			\begin{itemize}
			\smaller
				\item \texttt{customizeCsvHeader} - header aanpassen
				\item \texttt{customizeCsvRow} - een regel aanpassen
			\end{itemize}

		\item Voorbeeld:

			\begin{lstlisting}
use \TYPO3\CMS\Recordlist\RecordList\DatabaseRecordList;

$hookName = DatabaseRecordList::class;
$GLOBALS['TYPO3_CONF_VARS']['SC_OPTIONS'][$hookName]['customizeCsvRow'][] =
  \Vendor\ExtName\Hooks\CsvTest::class . '->customizeCsvRow';
$GLOBALS['TYPO3_CONF_VARS']['SC_OPTIONS'][$hookName]['customizeCsvHeader'][] =
  \Vendor\ExtName\Hooks\CsvTest::class . '->customizeCsvHeader';
\end{lstlisting}

	\end{itemize}

\end{frame}

% ------------------------------------------------------------------------------
% #85828 - Move symfony expression language handling into EXT:core

\begin{frame}[fragile]
	\frametitle{Wijzigingen voor ontwikkelaars}
	\framesubtitle{Symfony Expression Language}

	% decrease font size for code listing
	\lstset{basicstyle=\tiny\ttfamily}

	\begin{itemize}
		\item De \href{https://symfony.com/doc/current/components/expression_language/syntax.html}{Symfony Expression Language}
			is verplaatst van \texttt{EXT:form} naar de TYPO3 core
		\item Hierdoor is de expressietaal nu ook beschikbaar in andere delen
		\item De TYPO3 core biedt de class \texttt{DefaultProvider}, die direct gebruikt kan worden
			(zie voorbeeld hieronder) en afleidingen kunnen class\texttt{AbstractProvider} uitbreiden)

			\begin{lstlisting}
use \TYPO3\CMS\Core\ExpressionLanguage\DefaultProvider;
use \TYPO3\CMS\Core\ExpressionLanguage\Resolver;

$provider = GeneralUtility::makeInstance(DefaultProvider::class);
$conditionResolver = GeneralUtility::makeInstance(Resolver::class, $provider);
$conditionResolver->evaluate('1 < 2'); // result is true
			\end{lstlisting}

	\end{itemize}

\end{frame}

% ------------------------------------------------------------------------------
% #85356 - Add support to CurrencyViewHelper for mdash

\begin{frame}[fragile]
	\frametitle{Wijzigingen voor ontwikkelaars}
	\framesubtitle{"Munteenheid" ViewHelper}

%	% decrease font size for code listing
%	\lstset{basicstyle=\tiny\ttfamily}

	\begin{itemize}
		\item Een liggend streepje kan gebruikt worden in plaats van
			decimaal \texttt{00} in de Currency ViewHelper
		\item Optie \texttt{useDash="1"} schakelt dit in
		\item Het volgende voorbeeld produceert \texttt{123.-}

			\begin{lstlisting}
<f:format.currency useDash="1">123.00</f:format.currency>
			\end{lstlisting}

	\end{itemize}

\end{frame}

% ------------------------------------------------------------------------------
% #85699 - Deprecate methods in PageRepository
% #85727 - Move DatabaseIntegrityCheck to EXT:lowlevel

\begin{frame}[fragile]
	\frametitle{Wijzigingen voor ontwikkelaars}
	\framesubtitle{Wijzigingen in class \texttt{PageRepository}}

	\begin{itemize}
		\item De volgende methodes zijn als \textit{internal} gemarkeerd:

			\begin{itemize}
			\smaller
				\item \texttt{TYPO3\textbackslash
					CMS\textbackslash
					Frontend\textbackslash
					Page\textbackslash
					PageRepository::getMovePlaceholder()}

				\item \texttt{TYPO3\textbackslash
					CMS\textbackslash
					Frontend\textbackslash
					Page\textbackslash
					PageRepository::movePlhOL()}
			\end{itemize}

		\item De volgende class is verplaatst van de core naar de systeemextensie
			\texttt{lowlevel}:

			\begin{itemize}
			\smaller
				\item \texttt{TYPO3\textbackslash
					CMS\textbackslash
					Core\textbackslash
					Integrity\textbackslash
					DatabaseIntegrityCheck}
			\end{itemize}

	\end{itemize}

\end{frame}

% ------------------------------------------------------------------------------
% #85801 - Deprecate 2nd argument of GeneralUtility::explodeUrl2Array

\begin{frame}[fragile]
	\frametitle{Wijzigingen voor ontwikkelaars}
	\framesubtitle{Methode \texttt{GeneralUtility::explodeUrl2Array()}}

	\begin{itemize}
		\item Het tweede argument van methode
			\smaller
				\texttt{TYPO3\textbackslash
					CMS\textbackslash
					Core\textbackslash
					Utility\textbackslash
					GeneralUtility::explodeUrl2Array()}\newline
			\normalsize
			is als verouderd aangemerkt

		\item Indien gezet werd de string van het eerste argument omgezet in een
			multi-dimensionale array als rechte haken werden gebruikt in namen van
			variabelen.
		\item Dit kan versimpeld worden met de PHP functie \texttt{parse\_str()}:

			\begin{lstlisting}
$result = [];
parse_str($queryParametersAsString, $result);
			\end{lstlisting}

	\end{itemize}

\end{frame}

% ------------------------------------------------------------------------------
% #84280 - Make unit tests notice free

\begin{frame}[fragile]
	\frametitle{Wijzigingen voor ontwikkelaars}
	\framesubtitle{Unit-testen}

	\begin{itemize}
		\item De unit tests uit de core veroorzaken geen \texttt{E\_NOTICE} meer
		\item Hier is de vlag \texttt{\$suppressNotices = true} verwijderd en de versie
			van het test-framework verhoogd naar 4.4.0

	\end{itemize}

\end{frame}

% ------------------------------------------------------------------------------
