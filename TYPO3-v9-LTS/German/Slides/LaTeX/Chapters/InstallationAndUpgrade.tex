% ------------------------------------------------------------------------------
% TYPO3 v9 LTS - What's New (German Version)
%
% @license	Creative Commons BY-NC-SA 3.0
% @link		https://typo3.org/help/documentation/whats-new/
% @language	German
% ------------------------------------------------------------------------------

\section{Installation and Upgrade}
\begin{frame}[fragile]
	\frametitle{Installation und Upgrade}

	\begin{center}\huge{\color{typo3darkgrey}\textbf{Installation und Upgrade}}\end{center}
	\begin{center}\large{\textit{There is no better time to check out TYPO3 v9 LTS}}\end{center}

\end{frame}

% ------------------------------------------------------------------------------
% Classic Installation

\begin{frame}[fragile]
	\frametitle{Installation und Upgrade}
	\framesubtitle{Klassische Installationsmethode}

	\begin{itemize}
		\item Empfohlene \textit{klassische} Installationsschritte unter Linux/Mac OS X\newline
			(DocumentRoot ist beispielsweise \texttt{/var/www/site/htdocs}):
\begin{lstlisting}
$ cd /var/www/site/
$ wget --content-disposition get.typo3.org/9
$ tar xzf typo3_src-9.5.0.tar.gz
$ cd htdocs
$ ln -s ../typo3_src-9.5.0 typo3_src
$ ln -s typo3_src/index.php
$ ln -s typo3_src/typo3
$ touch FIRST_INSTALL
\end{lstlisting}

		\item Symbolic Links unter Microsoft Windows:

			\begin{itemize}
				\item unter Windows XP/2000 kann \texttt{junction} benutzt werden
				\item unter Windows Vista and Windows 7 oder höher kann \texttt{mklink} benutzt werden
			\end{itemize}

	\end{itemize}
\end{frame}

% ------------------------------------------------------------------------------
% Composer Installation

\begin{frame}[fragile]
	\frametitle{Installation und Upgrade}
	\framesubtitle{Installation mit \texttt{composer}}

	\begin{itemize}
		\item Installation mit \textit{composer} unter Linux, Mac OS X and Windows 10:

\begin{lstlisting}
$ cd /var/www/site/
$ composer create-project typo3/cms-base-distribution typo3 ^9
\end{lstlisting}

		\item Alternativ kann man eine benutzerdefinierte \texttt{composer.json} Datei erstellen und ausführen:

\begin{lstlisting}
$ composer install
\end{lstlisting}

			Weitere \texttt{composer.json} Beispielsdateien können unter \newline
			\smaller
				\href{https://composer.typo3.org}{https://composer.typo3.org} heruntergeladen werden
			\normalsize

	\end{itemize}
\end{frame}

% ------------------------------------------------------------------------------
% Upgrade to TYPO3 v9 LTS

\begin{frame}[fragile]
	\frametitle{Installation und Upgrade}
	\framesubtitle{Upgrade zu TYPO3 v9 LTS}

	\begin{itemize}
		\item Upgrade ist nur möglich von TYPO3 v8 LTS
		\item TYPO3 CMS < v8 LTS sollte zuerst auf TYPO3 CMS 8.7 aktualisiert werden
	\end{itemize}

	\begin{itemize}

		\item Upgrade-Anleitung:\newline
			\smaller\url{https://wiki.typo3.org/Upgrade#Upgrading_to_9.5_Long_Term_Support}\normalsize
		\item Officieller TYPO3 Leitfaden "TYPO3 Installation and Upgrading":
			\smaller\url{https://docs.typo3.org/typo3cms/InstallationGuide}\normalsize
		\item Generelles Vorgehen:
			\begin{itemize}
				\item Prüfen, ob Mindestvoraussetzungen erfüllt sind \small(PHP, MySQL, etc.)
				\item Das \textbf{deprecation\_*.log} der TYPO3 Instanz durchsehen
				\item Sämtliche Extensions auf den aktuellsten Stand bringen
				\item Neuen TYPO3 Quellcode entpacken und im Install Tool den Upgrade Wizard durchführen
			\end{itemize}
	\end{itemize}

\end{frame}

% ------------------------------------------------------------------------------
