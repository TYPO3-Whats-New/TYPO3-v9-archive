% ------------------------------------------------------------------------------
% TYPO3 Version 9.1 - What's New - Chapter "Changes for Developers" (Dutch Version)
%
% @author	Michael Schams <schams.net>
% @license	Creative Commons BY-NC-SA 3.0
% @link		http://typo3.org/download/release-notes/whats-new/
% @language	English
% ------------------------------------------------------------------------------
% LTXE-CHAPTER-UID:		846d40ce-66fdc2ec-750dcf95-33ce93e0
% LTXE-CHAPTER-NAME:	Changes for Developers
% ------------------------------------------------------------------------------

\section{Wijzigingen voor ontwikkelaars}
\begin{frame}[fragile]
	\frametitle{Wijzigingen voor ontwikkelaars}

	\begin{center}\huge{Hoofdstuk 3:}\end{center}
	\begin{center}\huge{\color{typo3darkgrey}\textbf{Wijzigingen voor ontwikkelaars}}\end{center}

\end{frame}

% ------------------------------------------------------------------------------
% LTXE-SLIDE-START
% LTXE-SLIDE-UID:		030ae153-ae8285a9-26aac067-e021ac81
% LTXE-SLIDE-ORIGIN:	d71432e3-7bb8957c-e2bb6e80-bc8b8c9b English
% LTXE-SLIDE-TITLE:		Add recursive filtering of arrays
% LTXE-SLIDE-REFERENCE:	83350-AddRecursiveArrayFiltering
% ------------------------------------------------------------------------------

\begin{frame}[fragile]
	\frametitle{Wijzigingen voor ontwikkelaars}
	\framesubtitle{\texttt{filterRecursive()}}

	\begin{itemize}
		\item Klasse
			\texttt{TYPO3\textbackslash
				CMS\textbackslash
				Core\textbackslash
				Utility\textbackslash
				ArrayUtility}\newline
			heeft een nieuwe functie om multidemensionale array's the filteren:\newline
			\texttt{filterRecursive()}
		\item De functie werkt net zoals de PHP functie \texttt{array\_filter()}\newline
			\small
				\href{https://php.net/manual/en/function.array-filter.php}{https://php.net/manual/en/function.array-filter.php}
			\normalsize
		\item Als geen callback is opgegeven dan worden waardes verwijderd als ze gelijk zijn aan de boolean waarde "\texttt{false}"
	\end{itemize}

\end{frame}

% ------------------------------------------------------------------------------
% LTXE-SLIDE-START
% LTXE-SLIDE-UID:		7bd97428-6bf83082-4a5a1daa-132f9fe8
% LTXE-SLIDE-ORIGIN:	15e90c1f-404b54ef-34ea893a-a0caa659 English
% LTXE-SLIDE-TITLE:		Feature Toggles (1)
% LTXE-SLIDE-REFERENCE:	83429-FeatureToggles
% ------------------------------------------------------------------------------

\begin{frame}[fragile]
	\frametitle{Wijzigingen voor ontwikkelaars}
	\framesubtitle{Functieschakelaars (1)}

	\begin{itemize}
		\item Een nieuwe API \textbf{Functieschakelaars} is toegevoegd
		\item Het doel van deze API is om alternatieve functies beter te ondersteunen,
			terwijl de oude beschikbaar blijven
		\item De API controleert een optie in de globale array \newline
			\small
				\texttt{\$TYPO3\_CONF\_VARS['SYS']['features']}
			\normalsize
		\item Zowel de TYPO3 core als extensies kunnen dan de alternatieve functionaliteit voor een functie bieden
		\item Typische voorbeelden voor het gebruik:
			\smaller
			\begin{itemize}
				\item Produceer exceptions in sommige gevallen in plaats van
					een tekst of foutmelding terug te geven.
				\item Schakel oude functionaliteit, die misschien gebruikt wordt, uit omdat\newline
					het systeem trager wordt.
				\item Schakel alternatieve PageNotFound afhandeling in voor een installatie.
			\end{itemize}
			\normalsize

	\end{itemize}

\end{frame}

% ------------------------------------------------------------------------------
% LTXE-SLIDE-START
% LTXE-SLIDE-UID:		808c64bd-31e2bc2e-a4585b41-74f44597
% LTXE-SLIDE-ORIGIN:	01a0a341-376f611a-ab4b764e-4ff86727 English
% LTXE-SLIDE-TITLE:		Feature Toggles (2)
% LTXE-SLIDE-REFERENCE:	83429-FeatureToggles
% ------------------------------------------------------------------------------

\begin{frame}[fragile]
	\frametitle{Wijzigingen voor ontwikkelaars}
	\framesubtitle{Functieschakelaars (2)}

	% decrease font size for code listing
	\lstset{basicstyle=\tiny\ttfamily}

	\begin{itemize}

		\item Functies zijn gedocumenteerd voor de TYPO3 core\newline
			\small
				(link toevoegen)
			\normalsize

		\item Extensieschrijvers kunnen de API gebruiken voor elke maatwerkfunctie van
			de extensie:

			\begin{lstlisting}
				if (GeneralUtility::makeInstance(Features::class)->isFeatureEnabled('mijnFunctienaam')) {
				  // maatwerk afhandeling
				}
			\end{lstlisting}

	\end{itemize}

\end{frame}

% ------------------------------------------------------------------------------
% LTXE-SLIDE-START
% LTXE-SLIDE-UID:		82101bf0-8baf8ce6-d71d0859-4958bc8b
% LTXE-SLIDE-ORIGIN:	865e2894-3ce87c06-0f92cc8b-b880318e English
% LTXE-SLIDE-TITLE:		Additional Hook For Record List
% LTXE-SLIDE-REFERENCE:	61170-AddAdditionalHookForRecordList
% ------------------------------------------------------------------------------

\begin{frame}[fragile]
	\frametitle{Wijzigingen voor ontwikkelaars}
	\framesubtitle{"Draw Header" Hook}

	% decrease font size for code listing
	\lstset{basicstyle=\tiny\ttfamily}

	\begin{itemize}
		\item Een nieuwe hook is toegevoegd aan \texttt{EXT:recordlist}
			om content boven alle andere content af te beelden

		\item Hook registreren:

			\begin{lstlisting}
				$GLOBALS['TYPO3_CONF_VARS']['SC_OPTIONS']['cms/layout/db_layout.php']
				  ['drawHeaderHook']['sys_note'] = \Vendor\Extkey\Hooks\PageHook::class . '->render';
			\end{lstlisting}

	\end{itemize}

\end{frame}

% ------------------------------------------------------------------------------
% LTXE-SLIDE-START
% LTXE-SLIDE-UID:		09b756dd-a92b9a9d-39c5b484-9ef98cd8
% LTXE-SLIDE-ORIGIN:	40b574bb-9f828b9d-cecafb06-096f4f29 English
% LTXE-SLIDE-TITLE:		BE User Login Hook (1)
% LTXE-SLIDE-REFERENCE:	83529-ExecuteHooksOnBackendUserLogin
% ------------------------------------------------------------------------------

\begin{frame}[fragile]
	\frametitle{Wijzigingen voor ontwikkelaars}
	\framesubtitle{Hook bij aanmelden BE gebruiker (1)}

	% decrease font size for code listing
	\lstset{basicstyle=\tiny\ttfamily}

	\begin{itemize}
		\item Bij het aanmelden van een backendgebruiker worden geregistreede hooks uitgevoerd
		\item Hiermee kunnen TYPO3 ontwikkelaars functies maken die
			\textit{iets} kunnen doen als een BE gebruiker zich aanmeldt

		\item Er kunnen bijvoorbeeld meldingen verstuurd worden:

			\begin{itemize}
				\item Plaats een bericht in Slack of een ander berichtensysteem.
				\item Stuur een SMS naar de mobiel van de gebruiker.
				\item Stuur een melding naar een systeem om verdachte activiteiten te monitoren.
				\item etc.
			\end{itemize}

	\end{itemize}

\end{frame}

% ------------------------------------------------------------------------------
% LTXE-SLIDE-START
% LTXE-SLIDE-UID:		d8557e9f-4190ea09-ba06f4cd-574391a8
% LTXE-SLIDE-ORIGIN:	8ab4904f-865ba977-d1d3cbd0-ff9b3797 English
% LTXE-SLIDE-TITLE:		BE User Login Hook (2)
% LTXE-SLIDE-REFERENCE:	83529-ExecuteHooksOnBackendUserLogin
% ------------------------------------------------------------------------------

\begin{frame}[fragile]
	\frametitle{Wijzigingen voor ontwikkelaars}
	\framesubtitle{Hook bij aanmelden BE gebruiker (2)}

	% decrease font size for code listing
	\lstset{basicstyle=\tiny\ttfamily}

	\begin{itemize}
		\item Hook registreren:

			\begin{lstlisting}
				$GLOBALS['TYPO3_CONF_VARS']['SC_OPTIONS']['t3lib/class.t3lib_userauthgroup.php']
				  ['backendUserLogin'][] = \Vendor\Extkey\Hooks\BackendUserLogin::class . '->dispatch';
			\end{lstlisting}

		\item Dit voert de functie \texttt{dispatch()} uit als een BE gebruiker zich aanmeldt en stuurt
			de gebruikersarray als parameter door naar de functie:

			\begin{lstlisting}
				public function dispatch($backendUser)
				{
				  if (isset($backendUser['user']['username'])) {
				    $username = $backendUser['user']['username'];
				    $email = $backendUser['user']['email'];
				    // doe iets...
				  }
				}
			\end{lstlisting}

	\end{itemize}

\end{frame}

% ------------------------------------------------------------------------------
