% ------------------------------------------------------------------------------
% TYPO3 Version 9.1 - What's New - Chapter "Deprecated Functions" (Dutch Version)
%
% @author	Michael Schams <schams.net>
% @license	Creative Commons BY-NC-SA 3.0
% @link		http://typo3.org/download/release-notes/whats-new/
% @language	English
% ------------------------------------------------------------------------------
% LTXE-CHAPTER-UID:		3f842373-9262b8d3-f9c8de76-cf29ce17
% LTXE-CHAPTER-NAME:	Deprecated Functions
% ------------------------------------------------------------------------------

\section{Verouderde/verwijderde functies}
\begin{frame}[fragile]
	\frametitle{Verouderde/verwijderde functies}

	\begin{center}\huge{Hoofdstuk 4:}\end{center}
	\begin{center}\huge{\color{typo3darkgrey}\textbf{Verouderde/verwijderde functies}}\end{center}

\end{frame}

% ------------------------------------------------------------------------------
% LTXE-SLIDE-START
% LTXE-SLIDE-UID:		4a362726-38cd8676-1b55976a-6de4e540
% LTXE-SLIDE-ORIGIN:	02a34a6e-add942aa-76ed434c-5651d314 English
% LTXE-SLIDE-TITLE:		Deprecated Usage of EXT:rsaauth
% LTXE-SLIDE-REFERENCE:	Deprecation-81852-DeprecatedUsageOfEXTrsaauth.rst
% ------------------------------------------------------------------------------

\begin{frame}[fragile]
	\frametitle{Verouderde/verwijderde functies}
	\framesubtitle{\texttt{EXT:rsaauth}}

	\begin{itemize}
		\item Extensie \texttt{EXT:rsaauth} is als \textbf{verouderd} aangemerkt.
		\item Door de toenemende acceptatie van SSL/TLS wordt de technologie van de extensie niet meer
			gezien als "veilig":

			\begin{itemize}
				\item Alleen het wachtwoord wordt versleuteld verstuurd
				\item Uitwisseling van de sleutel tussen server en browser is niet geauthentiseerd
					(man-in-the-middle aanval mogelijk)
				\item Sessie-ID's wordt onversleuteld verstuurd, maar zijn bijna net zo waardevol als
					wachtwoorden
			\end{itemize}

		\item Gebruik een beveiligde verbinding (HTTPS) in plaats hiervan en versleutel \underline{alle data}
			die tussen browser en server (TYPO3 frontend en backend) verstuurd wordt.

	\end{itemize}

	\smaller
		\textbf{NB: moderne browsers waarschuwen waneer gegevens onversleuteld verstuurd
			worden - niet alleen wachtwoorden of creditcard-gegevens.}
	\normalsize

\end{frame}

% ------------------------------------------------------------------------------
% LTXE-SLIDE-START
% LTXE-SLIDE-UID:		1f2ba8e4-f8cb87e2-d863cf1a-de46dc3a
% LTXE-SLIDE-ORIGIN:	ca23c0b4-6e8960b4-aede1a9f-0c84574c English
% LTXE-SLIDE-TITLE:		EXT:Form YAML Configurations Typoscript Option
% LTXE-SLIDE-REFERENCE:	Deprecation-82089-ExtFormYamlConfigurationsTyposcriptOption.rst
% ------------------------------------------------------------------------------
%
%\begin{frame}[fragile]
%	\frametitle{Verouderde/verwijderde functies}
%	\framesubtitle{YAML configuration of \texttt{EXT:form}}
%
%	% decrease font size for code listing
%	\lstset{basicstyle=\tiny\ttfamily}
%
%	\begin{itemize}
%		\item YAML configuration path has been marked as \textbf{deprecated}\newline
%			(will be removed in TYPO3 v10)
%
%			\begin{lstlisting}
%				plugin.tx_form {
%				  settings {
%				    yamlConfigurations {
%				      100 = EXT:my_site_package/Configuration/Yaml/CustomFormSetup.yaml
%				    }
%				  }
%				}
%			\end{lstlisting}
%
%		\item Instead, a single configuration file must be registered:
%
%			\begin{lstlisting}
%				plugin.tx_form {
%				  settings {
%				    configurationFile = EXT:my_site_package/Configuration/Yaml/CustomFormSetup.yaml
%				  }
%				}
%			\end{lstlisting}
%
%			\smaller
%				Same for backend setup (form editor): \texttt{module.tx\_form \{...\}}.
%			\normalsize
%
%	\end{itemize}
%
%\end{frame}

% ------------------------------------------------------------------------------
% LTXE-SLIDE-START
% LTXE-SLIDE-UID:		1118b4a8-ff2f7411-f8b8e3b9-4bcf1cdd
% LTXE-SLIDE-ORIGIN:	e82509d2-68646ba8-c887a63b-60ae97db English
% LTXE-SLIDE-TITLE:		Deprecate Unneeded RawValidator
% LTXE-SLIDE-REFERENCE:	83503-DeprecateUnneededRawValidator.rst
% ------------------------------------------------------------------------------

\begin{frame}[fragile]
	\frametitle{Verouderde/verwijderde functies}
	\framesubtitle{RawValidator}

	\begin{itemize}
		\item \texttt{RawValidator} is aangemerkt als \textbf{verouderd}
		\item Het was bedoeld als een soort NullObject om een "NoSuchValidatorException" te voorkomen,
			maar de exceptions worden afgevangen zodat de validator onnodig is
		\item Doordat de validator niets valideert is het waarschijnlijk dat deze wijziging geen
			invloed heeft op installaties
		\item Als ontwikkelaars de \texttt{RawValidator} gebruiken moeten ze er zelf een schrijven
	\end{itemize}


\end{frame}

% ------------------------------------------------------------------------------
