% ------------------------------------------------------------------------------
% TYPO3 Version 9.1 - What's New - Chapter "Changes for Developers" (Serbian Version)
%
% @author	Michael Schams <schams.net>
% @license	Creative Commons BY-NC-SA 3.0
% @link		http://typo3.org/download/release-notes/whats-new/
% @language	English
% ------------------------------------------------------------------------------
% LTXE-CHAPTER-UID:		846d40ce-66fdc2ec-750dcf95-33ce93e0
% LTXE-CHAPTER-NAME:	Changes for Developers
% ------------------------------------------------------------------------------

\section{Izmene za programere}
\begin{frame}[fragile]
	\frametitle{Izmene za programere}

	\begin{center}\huge{Poglavlje 3:}\end{center}
	\begin{center}\huge{\color{typo3darkgrey}\textbf{Izmene za programere}}\end{center}

\end{frame}

% ------------------------------------------------------------------------------
% LTXE-SLIDE-START
% LTXE-SLIDE-UID:		53ac46be-2d21b283-4db6d9df-a1217eeb
% LTXE-SLIDE-ORIGIN:	d71432e3-7bb8957c-e2bb6e80-bc8b8c9b English
% LTXE-SLIDE-TITLE:		Add recursive filtering of arrays
% LTXE-SLIDE-REFERENCE:	83350-AddRecursiveArrayFiltering
% ------------------------------------------------------------------------------

\begin{frame}[fragile]
	\frametitle{Izmene za programere}
	\framesubtitle{\texttt{filterRecursive()}}

	\begin{itemize}
		\item Klasa
			\texttt{TYPO3\textbackslash
				CMS\textbackslash
				Core\textbackslash
				Utility\textbackslash
				ArrayUtility}\newline
			je dobila novi metod za filtriranje visedimenzionih nizeva:\newline
			\texttt{filterRecursive()}
		\item Ovaj metod se ponasa kao i PHP funkcija \texttt{array\_filter()}\newline
			\small
				\href{https://php.net/manual/en/function.array-filter.php}{https://php.net/manual/en/function.array-filter.php}
			\normalsize
		\item Ako nije definisan callback uklanjaju su vrednosti koje imaju vrednost jadnaku vrednosti boolean "\texttt{false}"
	\end{itemize}

\end{frame}

% ------------------------------------------------------------------------------
% LTXE-SLIDE-START
% LTXE-SLIDE-UID:		b4d4881c-d292b801-9319a2cb-f1430022
% LTXE-SLIDE-ORIGIN:	15e90c1f-404b54ef-34ea893a-a0caa659 English
% LTXE-SLIDE-TITLE:		Feature Toggles (1)
% LTXE-SLIDE-REFERENCE:	83429-FeatureToggles
% ------------------------------------------------------------------------------

\begin{frame}[fragile]
	\frametitle{Izmene za programere}
	\framesubtitle{Feature Toggles (1)}

	\begin{itemize}
		\item Novi API \textbf{Feature Toggles} je implementiran
		\item Svrha ovog API-ja je da bolje podrzi alternativne funkcionalnosti,
			a da pri tome zadrzi stare funkcionalnosti
		\item API proverava ceo sistem unutar niza\newline
			\small
				\texttt{\$TYPO3\_CONF\_VARS['SYS']['features']}
			\normalsize
		\item TYPO3 jezgro kao i ekstenzije mogu da pruze alternativne funkcionalnosti za
			odredjena svojstva
		\item Tipican primer je:
			\smaller
			\begin{itemize}
				\item Baciti izuzetak u odredjenim okolnostima umesto vracanja
					stringa kao greske.
				\item Iskljuciti zastarele funkcionalnosti, koje mogu i dalje da se koriste,
					ali usporavaju sistem.
				\item Omoguciti alternativno upravljanje PageNotFound za instalaciju.
			\end{itemize}
			\normalsize

	\end{itemize}

\end{frame}

% ------------------------------------------------------------------------------
% LTXE-SLIDE-START
% LTXE-SLIDE-UID:		06ce8a20-417b0e63-984ba574-521297f5
% LTXE-SLIDE-ORIGIN:	01a0a341-376f611a-ab4b764e-4ff86727 English
% LTXE-SLIDE-TITLE:		Feature Toggles (2)
% LTXE-SLIDE-REFERENCE:	83429-FeatureToggles
% ------------------------------------------------------------------------------

\begin{frame}[fragile]
	\frametitle{Izmene za programere}
	\framesubtitle{Feature Toggles (2)}

	% decrease font size for code listing
	\lstset{basicstyle=\tiny\ttfamily}

	\begin{itemize}

		\item "Features" za TYPO3 jezgro su dokumentovane

		\item Autori prosirenja mogu da koriste API za bilo koju karakteristiku u ekstenziji:

			\begin{lstlisting}
				if (GeneralUtility::makeInstance(Features::class)->isFeatureEnabled('myFeatureName')) {
				  // do custom processing
				}
			\end{lstlisting}

	\end{itemize}

\end{frame}

% ------------------------------------------------------------------------------
% LTXE-SLIDE-START
% LTXE-SLIDE-UID:		142cf616-8e960b44-5d2bd9b4-1112553b
% LTXE-SLIDE-ORIGIN:	865e2894-3ce87c06-0f92cc8b-b880318e English
% LTXE-SLIDE-TITLE:		Additional Hook For Record List
% LTXE-SLIDE-REFERENCE:	61170-AddAdditionalHookForRecordList
% ------------------------------------------------------------------------------

\begin{frame}[fragile]
	\frametitle{Izmene za programere}
	\framesubtitle{"Draw Header" Hook}

	% decrease font size for code listing
	\lstset{basicstyle=\tiny\ttfamily}

	\begin{itemize}
		\item Novi hook je dodat u \texttt{EXT:recordlist}
			sa svrhom da izrendera sadrzaj iznad bilo kog drugog sadrzaja
			
		\item Registrovanje hook-a:

			\begin{lstlisting}
				$GLOBALS['TYPO3_CONF_VARS']['SC_OPTIONS']['cms/layout/db_layout.php']
				  ['drawHeaderHook']['sys_note'] = \Vendor\Extkey\Hooks\PageHook::class . '->render';
			\end{lstlisting}

	\end{itemize}

\end{frame}

% ------------------------------------------------------------------------------
% LTXE-SLIDE-START
% LTXE-SLIDE-UID:		c1686a6a-3a7fd7bd-dcc5dec0-c1db722b
% LTXE-SLIDE-ORIGIN:	40b574bb-9f828b9d-cecafb06-096f4f29 English
% LTXE-SLIDE-TITLE:		BE User Login Hook (1)
% LTXE-SLIDE-REFERENCE:	83529-ExecuteHooksOnBackendUserLogin
% ------------------------------------------------------------------------------

\begin{frame}[fragile]
	\frametitle{Izmene za programere}
	\framesubtitle{BE User Login Hook (1)}

	% decrease font size for code listing
	\lstset{basicstyle=\tiny\ttfamily}

	\begin{itemize}
		\item Registrovani hook-ovi se izvrsavaju prilikom prijave na administratorski interfejs
		\item Ovo omogucava TYPO3 programerima da naprave funkcije koje rade 
			\textit{nesto} kada se BE korisnik loguje.

		\item Servisi za obavestavanje su tipican primer:

			\begin{itemize}
				\item Slanje poruke na Slack ili slican sistem za poruke.
				\item Slanje tekstualne poruke (SMS) na broj korisnikovog mobilnog telefona.
				\item Prosledjivanje ovog dogadjaja drugim sistemima za pracenje sumljivih aktivnosti.
				\item itd.
			\end{itemize}

	\end{itemize}

\end{frame}

% ------------------------------------------------------------------------------
% LTXE-SLIDE-START
% LTXE-SLIDE-UID:		872c27cc-bbf60e39-79e1c718-53393df5
% LTXE-SLIDE-ORIGIN:	8ab4904f-865ba977-d1d3cbd0-ff9b3797 English
% LTXE-SLIDE-TITLE:		BE User Login Hook (2)
% LTXE-SLIDE-REFERENCE:	83529-ExecuteHooksOnBackendUserLogin
% ------------------------------------------------------------------------------

\begin{frame}[fragile]
	\frametitle{Izmene za programere}
	\framesubtitle{BE User Login Hook (2)}

	% decrease font size for code listing
	\lstset{basicstyle=\tiny\ttfamily}

	\begin{itemize}
		\item Registrovanje hook-a:

			\begin{lstlisting}
				$GLOBALS['TYPO3_CONF_VARS']['SC_OPTIONS']['t3lib/class.t3lib_userauthgroup.php']
				  ['backendUserLogin'][] = \Vendor\Extkey\Hooks\BackendUserLogin::class . '->dispatch';
			\end{lstlisting}

		\item Ovaj kod izvrsava metod \texttt{dispatch()} kada se BE korisnik uloguje
			i prosledjuje korisnicki niz kao parametar metoda

			\begin{lstlisting}
				public function dispatch($backendUser)
				{
				  if (isset($backendUser['user']['username'])) {
				    $username = $backendUser['user']['username'];
				    $email = $backendUser['user']['email'];
				    // do something...
				  }
				}
			\end{lstlisting}

	\end{itemize}

\end{frame}

% ------------------------------------------------------------------------------
