% ------------------------------------------------------------------------------
% TYPO3 Version 9.1 - What's New - Chapter "Deprecated Functions" (Serbian Version)
%
% @author	Michael Schams <schams.net>
% @license	Creative Commons BY-NC-SA 3.0
% @link		http://typo3.org/download/release-notes/whats-new/
% @language	English
% ------------------------------------------------------------------------------
% LTXE-CHAPTER-UID:		3f842373-9262b8d3-f9c8de76-cf29ce17
% LTXE-CHAPTER-NAME:	Deprecated Functions
% ------------------------------------------------------------------------------

\section{Zastarele/Izbacene Funkcije}
\begin{frame}[fragile]
	\frametitle{Zastarele/Izbacene Funkcije}

	\begin{center}\huge{Poglavlje 4:}\end{center}
	\begin{center}\huge{\color{typo3darkgrey}\textbf{Zastarele/Izbacene Funkcije}}\end{center}

\end{frame}

% ------------------------------------------------------------------------------
% LTXE-SLIDE-START
% LTXE-SLIDE-UID:		12381f23-f557f92b-f9d980b5-ae0c7fd9
% LTXE-SLIDE-ORIGIN:	02a34a6e-add942aa-76ed434c-5651d314 English
% LTXE-SLIDE-TITLE:		Deprecated Usage of EXT:rsaauth
% LTXE-SLIDE-REFERENCE:	Deprecation-81852-DeprecatedUsageOfEXTrsaauth.rst
% ------------------------------------------------------------------------------

\begin{frame}[fragile]
	\frametitle{Zastarele/Izbacene Funkcije}
	\framesubtitle{\texttt{EXT:rsaauth}}

	\begin{itemize}
		\item Ekstenzija \texttt{EXT:rsaauth} je oznacena kao \textbf{zastarela}
		\item Zbog brzog prihvatanja SSL/TLS, tehnologija koju koristi ova ekstenzija 
			vise se ne smatra "sigurnom":

			\begin{itemize}
				\item Samo se lozinka prenosi enkriptovana
				\item Razmena kljuceva izmedju servera i klijenta nije autentifikovana\newline
					(dozvoljava man-in-the-middle napade)
				\item ID-jevi sesije se prenose neenkriptovani, a vazni su skoro kao i lozinke
			\end{itemize}

		\item Umesto ovoga koristiti sigurnu konekciju (HTTPS) i enkriptovati \underline{sve podatke}
			koji se salju izmedju servera i klijenta (TYPO3 korisnicki interfejs i administratorski interfejs)

	\end{itemize}

	\smaller
		\textbf{Napomena: moderni pretrazivaci upozoravaju korisnike kada su podaci sa forme poslati 
			putem neenkriptovane konekcije, ne samo za lozinke i podatke kreditnih kartica}
	\normalsize

\end{frame}

% ------------------------------------------------------------------------------
% LTXE-SLIDE-START
% LTXE-SLIDE-UID:		07cb7b85-a1d24afd-362866ea-f61aaf4d
% LTXE-SLIDE-ORIGIN:	ca23c0b4-6e8960b4-aede1a9f-0c84574c English
% LTXE-SLIDE-TITLE:		EXT:Form YAML Configurations Typoscript Option
% LTXE-SLIDE-REFERENCE:	Deprecation-82089-ExtFormYamlConfigurationsTyposcriptOption.rst
% ------------------------------------------------------------------------------
%
%\begin{frame}[fragile]
%	\frametitle{Zastarele/Izbacene Funkcije}
%	\framesubtitle{YAML configuration of \texttt{EXT:form}}
%
%	% decrease font size for code listing
%	\lstset{basicstyle=\tiny\ttfamily}
%
%	\begin{itemize}
%		\item YAML configuration path has been marked as \textbf{deprecated}\newline
%			(will be removed in TYPO3 v10)
%
%			\begin{lstlisting}
%				plugin.tx_form {
%				  settings {
%				    yamlConfigurations {
%				      100 = EXT:my_site_package/Configuration/Yaml/CustomFormSetup.yaml
%				    }
%				  }
%				}
%			\end{lstlisting}
%
%		\item Instead, a single configuration file must be registered:
%
%			\begin{lstlisting}
%				plugin.tx_form {
%				  settings {
%				    configurationFile = EXT:my_site_package/Configuration/Yaml/CustomFormSetup.yaml
%				  }
%				}
%			\end{lstlisting}
%
%			\smaller
%				Same for backend setup (form editor): \texttt{module.tx\_form \{...\}}.
%			\normalsize
%
%	\end{itemize}
%
%\end{frame}

% ------------------------------------------------------------------------------
% LTXE-SLIDE-START
% LTXE-SLIDE-UID:		761d6d9f-adb5aa47-3af50e57-4f2be67c
% LTXE-SLIDE-ORIGIN:	e82509d2-68646ba8-c887a63b-60ae97db English
% LTXE-SLIDE-TITLE:		Deprecate Unneeded RawValidator
% LTXE-SLIDE-REFERENCE:	83503-DeprecateUnneededRawValidator.rst
% ------------------------------------------------------------------------------

\begin{frame}[fragile]
	\frametitle{Zastarele/Izbacene Funkcije}
	\framesubtitle{RawValidator}

	\begin{itemize}
		\item \texttt{RawValidator} je oznacen kao  \textbf{zastareo}
		\item Zamisljen je kao neka vrsta NullObject-a da spreci "NoSuchValidatorException",
			ali ovi izuzetci su uhvaceni sto cini ovaj validator zastarelim
		\item Iz razloga sto validator ne validira nista,
			velika je verovatnoca da ova izmena nece uticati ni na jednu instalaciju.
		\item U slucaju da programeri ipak koriste \texttt{RawValidator}, morace sami da ga implementiraju
	\end{itemize}


\end{frame}

% ------------------------------------------------------------------------------
