% ------------------------------------------------------------------------------
% TYPO3 Version 9.1 - What's New - Chapter "Deprecated Functions" (English Version)
%
% @author	Michael Schams <schams.net>
% @license	Creative Commons BY-NC-SA 3.0
% @link		http://typo3.org/download/release-notes/whats-new/
% @language	English
% ------------------------------------------------------------------------------
% LTXE-CHAPTER-UID:		3f842373-9262b8d3-f9c8de76-cf29ce17
% LTXE-CHAPTER-NAME:	Deprecated Functions
% ------------------------------------------------------------------------------

\section{Veraltete/Entfernte Funktionen}
\begin{frame}[fragile]
	\frametitle{Veraltete/Entfernte Funktionen}

	\begin{center}\huge{Kapitel 4:}\end{center}
	\begin{center}\huge{\color{typo3darkgrey}\textbf{Veraltete/Entfernte Funktionen}}\end{center}

\end{frame}

% ------------------------------------------------------------------------------
% LTXE-SLIDE-START
% LTXE-SLIDE-UID:		02a34a6e-add942aa-76ed434c-5651d314
% LTXE-SLIDE-TITLE:		Deprecated Usage of EXT:rsaauth
% LTXE-SLIDE-REFERENCE:	Deprecation-81852-DeprecatedUsageOfEXTrsaauth.rst
% ------------------------------------------------------------------------------

\begin{frame}[fragile]
	\frametitle{Veraltete/Entfernte Funktionen}
	\framesubtitle{\texttt{EXT:rsaauth}}

	\begin{itemize}
		\item Die Erweiterung \texttt{EXT:rsaauth} wurde als \textbf{veraltet} markiert
		\item Aufgrund der schnell wachsenden Akzeptanz von SSL/TLS, wird die benutzte
			Technologie nicht mehr als "sicher" betrachtet:

			\begin{itemize}
				\item Nur das Passwort wird verschlüsselt übertragen
				\item Der Schlüsselaustausch zwischen Client und Server wird nicht authentifiziert
					(jenes ermöglicht "man-in-the-middle"-Angriffe)
				\item Session IDs werden unverschlüsselt übertragen, sind aber fast so wertvoll wie
					Kennwörter
			\end{itemize}

		\item Stattdessen sollte generell HTTPS verwendet und somit \underline{alle Daten}
			zwischen Client und Server verschlüsselt werden (FE und BE)

	\end{itemize}

	\smaller
		\textbf{Hinweis: moderne Browser warnen Benutzer standardmäßig, wenn Formulardaten über eine
		   unverschlüsselte Verbindung übermittelt werden - nicht nur Passwörter oder Kreditkartendaten.}
	\normalsize

\end{frame}

% ------------------------------------------------------------------------------
% LTXE-SLIDE-START
% LTXE-SLIDE-UID:		ca23c0b4-6e8960b4-aede1a9f-0c84574c
% LTXE-SLIDE-TITLE:		EXT:Form YAML Configurations Typoscript Option
% LTXE-SLIDE-REFERENCE:	Deprecation-82089-ExtFormYamlConfigurationsTyposcriptOption.rst
% ------------------------------------------------------------------------------
%
%\begin{frame}[fragile]
%	\frametitle{Deprecated/Removed Functions}
%	\framesubtitle{YAML configuration of \texttt{EXT:form}}
%
%	% decrease font size for code listing
%	\lstset{basicstyle=\tiny\ttfamily}
%
%	\begin{itemize}
%		\item YAML configuration path has been marked as \textbf{deprecated}\newline
%			(will be removed in TYPO3 v10)
%
%			\begin{lstlisting}
%				plugin.tx_form {
%				  settings {
%				    yamlConfigurations {
%				      100 = EXT:my_site_package/Configuration/Yaml/CustomFormSetup.yaml
%				    }
%				  }
%				}
%			\end{lstlisting}
%
%		\item Instead, a single configuration file must be registered:
%
%			\begin{lstlisting}
%				plugin.tx_form {
%				  settings {
%				    configurationFile = EXT:my_site_package/Configuration/Yaml/CustomFormSetup.yaml
%				  }
%				}
%			\end{lstlisting}
%
%			\smaller
%				Same for backend setup (form editor): \texttt{module.tx\_form \{...\}}.
%			\normalsize
%
%	\end{itemize}
%
%\end{frame}

% ------------------------------------------------------------------------------
% LTXE-SLIDE-START
% LTXE-SLIDE-UID:		e82509d2-68646ba8-c887a63b-60ae97db
% LTXE-SLIDE-TITLE:		Deprecate Unneeded RawValidator
% LTXE-SLIDE-REFERENCE:	83503-DeprecateUnneededRawValidator.rst
% ------------------------------------------------------------------------------

\begin{frame}[fragile]
	\frametitle{Veraltete/Entfernte Funktionen}
	\framesubtitle{RawValidator}

	\begin{itemize}
		\item \texttt{RawValidator} wurde als \textbf{veraltet} markiert
		\item Es sollte eine Art NullObject sein, um eine "NoSuchValidatorException" zu verhindern,
			aber diese Exceptions werden abgefangen, wodurch der Validator obsolet wird
		\item Aufgrund der Tatsache, dass der Validator nichts validiert,
			ist die Chance groß, dass diese Änderung keine Installationen beeinflusst
		\item Falls Entwickler \texttt{RawValidator} verwenden, müssen sie es selbst implementieren
	\end{itemize}


\end{frame}

% ------------------------------------------------------------------------------
