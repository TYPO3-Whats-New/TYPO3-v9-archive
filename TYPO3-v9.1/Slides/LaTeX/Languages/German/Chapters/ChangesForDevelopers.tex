% ------------------------------------------------------------------------------
% TYPO3 Version 9.1 - What's New - Chapter "Changes for Developers" (English Version)
%
% @author	Michael Schams <schams.net>
% @license	Creative Commons BY-NC-SA 3.0
% @link		http://typo3.org/download/release-notes/whats-new/
% @language	English
% ------------------------------------------------------------------------------
% LTXE-CHAPTER-UID:		846d40ce-66fdc2ec-750dcf95-33ce93e0
% LTXE-CHAPTER-NAME:	Changes for Developers
% ------------------------------------------------------------------------------

\section{Änderungen für Entwickler}
\begin{frame}[fragile]
	\frametitle{Änderungen für Entwickler}

	\begin{center}\huge{Kapitel 3:}\end{center}
	\begin{center}\huge{\color{typo3darkgrey}\textbf{Änderungen für Entwickler}}\end{center}

\end{frame}

% ------------------------------------------------------------------------------
% LTXE-SLIDE-START
% LTXE-SLIDE-UID:		d71432e3-7bb8957c-e2bb6e80-bc8b8c9b
% LTXE-SLIDE-TITLE:		Add recursive filtering of arrays
% LTXE-SLIDE-REFERENCE:	83350-AddRecursiveArrayFiltering
% ------------------------------------------------------------------------------

\begin{frame}[fragile]
	\frametitle{Änderungen für Entwickler}
	\framesubtitle{\texttt{filterRecursive()}}

	\begin{itemize}
		\item Klasse
			\texttt{TYPO3\textbackslash
				CMS\textbackslash
				Core\textbackslash
				Utility\textbackslash
				ArrayUtility}\newline
			bietet eine neue Methode zum Filtern mehrdimensionaler Arrays:\newline
			\texttt{filterRecursive()}
		\item Diese Methode verhält sich genau wie die PHP-Funktion \texttt{array\_filter()}\newline
			\small
				\href{https://php.net/manual/en/function.array-filter.php}{https://php.net/manual/en/function.array-filter.php}
			\normalsize
		\item Wenn kein Callback definiert ist, werden Werte entfernt, wenn diese den Wert boolean "\texttt{false}" haben
	\end{itemize}

\end{frame}

% ------------------------------------------------------------------------------
% LTXE-SLIDE-START
% LTXE-SLIDE-UID:		15e90c1f-404b54ef-34ea893a-a0caa659
% LTXE-SLIDE-TITLE:		Feature Toggles (1)
% LTXE-SLIDE-REFERENCE:	83429-FeatureToggles
% ------------------------------------------------------------------------------

\begin{frame}[fragile]
	\frametitle{Änderungen für Entwickler}
	\framesubtitle{Feature Toggles (1)}

	\begin{itemize}
		\item Neue API \textbf{Feature Toggles} wurde implementiert
		\item Der Zweck dieser API ist, alternative Funktionalitäten
			unter Beibehaltung bestehender Funktionalitäten besser zu unterstützen. 
		\item API prüft auf ein systemweites Optionsfeld innerhalb von\newline
			\small
				\texttt{\$TYPO3\_CONF\_VARS['SYS']['features']}
			\normalsize
		\item Sowohl der TYPO3-Core als auch die Erweiterungen können dann
			alternative Funktionalität für eine bestimmte Funkion liefern
		\item Typische Anwendungsfälle sind z.B.:
			\smaller
			\begin{itemize}
				\item Unter bestimmten Umständen Exceptions zu generieren, anstatt
					Fehlermeldungen als String zurück zu geben.
				\item Veraltete Funkionenen zu deaktivieren, die noch benutzt werden könnten,
					aber das System verlangsamen.
				\item Alternative PageNotFound-Behandlung für eine Installation aktivieren.
			\end{itemize}
			\normalsize

	\end{itemize}

\end{frame}

% ------------------------------------------------------------------------------
% LTXE-SLIDE-START
% LTXE-SLIDE-UID:		01a0a341-376f611a-ab4b764e-4ff86727
% LTXE-SLIDE-TITLE:		Feature Toggles (2)
% LTXE-SLIDE-REFERENCE:	83429-FeatureToggles
% ------------------------------------------------------------------------------

\begin{frame}[fragile]
	\frametitle{Änderungen für Entwickler}
	\framesubtitle{Feature Toggles (2)}

	% decrease font size for code listing
	\lstset{basicstyle=\tiny\ttfamily}

	\begin{itemize}

		\item Geplant ist, "Features" des TYPO3 Cores entsprechend zu dokumentieren

		\item Die Verfasser der Erweiterungen können die API für jede benutzerdefinierte Funktion
			verwenden:

			\begin{lstlisting}
				if (GeneralUtility::makeInstance(Features::class)->isFeatureEnabled('myFeatureName')) {
				  // do custom processing
				}
			\end{lstlisting}

	\end{itemize}

\end{frame}

% ------------------------------------------------------------------------------
% LTXE-SLIDE-START
% LTXE-SLIDE-UID:		865e2894-3ce87c06-0f92cc8b-b880318e
% LTXE-SLIDE-TITLE:		Additional Hook For Record List
% LTXE-SLIDE-REFERENCE:	61170-AddAdditionalHookForRecordList
% ------------------------------------------------------------------------------

\begin{frame}[fragile]
	\frametitle{Änderungen für Entwickler}
	\framesubtitle{"Draw Header" Hook}

	% decrease font size for code listing
	\lstset{basicstyle=\tiny\ttfamily}

	\begin{itemize}
		\item Ein neuer Hook wurde zu \texttt{EXT:recordlist}hinzugefügt
			um Inhalte über jedem anderen Inhalt darzustellen

		\item Einen Hook registrieren:

			\begin{lstlisting}
				$GLOBALS['TYPO3_CONF_VARS']['SC_OPTIONS']['cms/layout/db_layout.php']
				  ['drawHeaderHook']['sys_note'] = \Vendor\Extkey\Hooks\PageHook::class . '->render';
			\end{lstlisting}

	\end{itemize}

\end{frame}

% ------------------------------------------------------------------------------
% LTXE-SLIDE-START
% LTXE-SLIDE-UID:		40b574bb-9f828b9d-cecafb06-096f4f29
% LTXE-SLIDE-TITLE:		BE User Login Hook (1)
% LTXE-SLIDE-REFERENCE:	83529-ExecuteHooksOnBackendUserLogin
% ------------------------------------------------------------------------------

\begin{frame}[fragile]
	\frametitle{Änderungen für Entwickler}
	\framesubtitle{BE User Login Hook (1)}

	% decrease font size for code listing
	\lstset{basicstyle=\tiny\ttfamily}

	\begin{itemize}
		\item Registrierte Hooks werden bei der Backend Benutzeranmeldungen ausgeführt
		\item Dadurch können TYPO3-Entwickler Funktionen erstellen, ausgeführt
			 werden, wenn ein BE-Benutzer sich anmeldet

		\item Benachrichtigungsdienste sind typische Anwendungsfälle:

			\begin{itemize}
				\item Eine Nachricht an Slack oder ein ähnliches Messaging-System posten.
				\item Eine SMS an die Handynummer des Benutzers senden.
				\item Dieses Ereignis an andere Systeme übergeben, um verdächtige
					Aktivitäten zu überwachen.
				\item etc.
			\end{itemize}

	\end{itemize}

\end{frame}

% ------------------------------------------------------------------------------
% LTXE-SLIDE-START
% LTXE-SLIDE-UID:		8ab4904f-865ba977-d1d3cbd0-ff9b3797
% LTXE-SLIDE-TITLE:		BE User Login Hook (2)
% LTXE-SLIDE-REFERENCE:	83529-ExecuteHooksOnBackendUserLogin
% ------------------------------------------------------------------------------

\begin{frame}[fragile]
	\frametitle{Änderungen für Entwickler}
	\framesubtitle{BE User Login Hook (2)}

	% decrease font size for code listing
	\lstset{basicstyle=\tiny\ttfamily}

	\begin{itemize}
		\item Registere einen Hook:

			\begin{lstlisting}
				$GLOBALS['TYPO3_CONF_VARS']['SC_OPTIONS']['t3lib/class.t3lib_userauthgroup.php']
				  ['backendUserLogin'][] = \Vendor\Extkey\Hooks\BackendUserLogin::class . '->dispatch';
			\end{lstlisting}

		\item Dadurch wird die \texttt{dispatch()} Methode ausgeführt, wenn sich ein Benutzer einloggt.
			Das Benutzer-Array als Parameter für die Methode übergeben:

			\begin{lstlisting}
				public function dispatch($backendUser)
				{
				  if (isset($backendUser['user']['username'])) {
				    $username = $backendUser['user']['username'];
				    $email = $backendUser['user']['email'];
				    // do something...
				  }
				}
			\end{lstlisting}

	\end{itemize}

\end{frame}

% ------------------------------------------------------------------------------
