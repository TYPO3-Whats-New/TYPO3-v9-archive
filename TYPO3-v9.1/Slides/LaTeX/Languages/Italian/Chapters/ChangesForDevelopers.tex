% ------------------------------------------------------------------------------
% TYPO3 Version 9.1 - What's New - Chapter "Changes for Developers" (Italian Version)
%
% @author	Michael Schams <schams.net>
% @license	Creative Commons BY-NC-SA 3.0
% @link		http://typo3.org/download/release-notes/whats-new/
% @language	English
% ------------------------------------------------------------------------------
% LTXE-CHAPTER-UID:		846d40ce-66fdc2ec-750dcf95-33ce93e0
% LTXE-CHAPTER-NAME:	Changes for Developers
% ------------------------------------------------------------------------------

\section{Modifiche per sviluppatori}
\begin{frame}[fragile]
	\frametitle{Modifiche per sviluppatori}

	\begin{center}\huge{Capitolo 3:}\end{center}
	\begin{center}\huge{\color{typo3darkgrey}\textbf{Modifiche per sviluppatori}}\end{center}

\end{frame}

% ------------------------------------------------------------------------------
% LTXE-SLIDE-START
% LTXE-SLIDE-UID:		f0454058-9ddee245-fee04f19-8dcaeedb
% LTXE-SLIDE-ORIGIN:	d71432e3-7bb8957c-e2bb6e80-bc8b8c9b English
% LTXE-SLIDE-TITLE:		Add recursive filtering of arrays
% LTXE-SLIDE-REFERENCE:	83350-AddRecursiveArrayFiltering
% ------------------------------------------------------------------------------

\begin{frame}[fragile]
	\frametitle{Modifiche per sviluppatori}
	\framesubtitle{\texttt{filterRecursive()}}

	\begin{itemize}
		\item La Class
			\texttt{TYPO3\textbackslash
				CMS\textbackslash
				Core\textbackslash
				Utility\textbackslash
				ArrayUtility}\newline
			presenta un nuovo metodo per filtrare gli array multidimensionali:\newline
			\texttt{filterRecursive()}
		\item Questo metodo si comporta come la funzione PHP \texttt{array\_filter()}\newline
			\small
				\href{https://php.net/manual/en/function.array-filter.php}{https://php.net/manual/en/function.array-filter.php}
			\normalsize
		\item Se non è definita una callback, quando i valori equivalgono al boolean "\texttt{false}" vengono rimossi
	\end{itemize}

\end{frame}

% ------------------------------------------------------------------------------
% LTXE-SLIDE-START
% LTXE-SLIDE-UID:		73ee1775-079dd1fc-e66954a2-fe6cf610
% LTXE-SLIDE-ORIGIN:	15e90c1f-404b54ef-34ea893a-a0caa659 English
% LTXE-SLIDE-TITLE:		Feature Toggles (1)
% LTXE-SLIDE-REFERENCE:	83429-FeatureToggles
% ------------------------------------------------------------------------------

\begin{frame}[fragile]
	\frametitle{Modifiche per sviluppatori}
	\framesubtitle{Attiva/Disattiva funzionalità [Feature Toggles] (1)}

	\begin{itemize}
		\item La nuova API \textbf{Feature Toggles} è stata implementata.
		\item L'obiettivo di questa API è quello di supportare al meglio le funzionalità alternative,
			pur mantenendo le vecchie funzionalità
		\item L'API verifica in un array di opzioni a livello di sistema\newline
			\small
				\texttt{\$TYPO3\_CONF\_VARS['SYS']['features']}
			\normalsize
		\item Sia il core di TYPO3 che le estensioni possono quindi fornire
			funzionalità alternative per una determinata azione
		\item Tipi casi d'uso per esempio:
			\smaller
			\begin{itemize}
				\item Creare eccezioni in determinate occasioni, invece di una stringa di errore.
				\item Disabilitare una funzionalità obsoleta, che potrebbe essere ancora utilizzata,
					ma rallenta il sistema.
				\item Abilita una funzione alternativa di gestione PageNotFound in un'installazione.
			\end{itemize}
			\normalsize

	\end{itemize}

\end{frame}

% ------------------------------------------------------------------------------
% LTXE-SLIDE-START
% LTXE-SLIDE-UID:		a7509650-cc2b9222-31f6ea94-d6cc2389
% LTXE-SLIDE-ORIGIN:	01a0a341-376f611a-ab4b764e-4ff86727 English
% LTXE-SLIDE-TITLE:		Feature Toggles (2)
% LTXE-SLIDE-REFERENCE:	83429-FeatureToggles
% ------------------------------------------------------------------------------

\begin{frame}[fragile]
	\frametitle{Modifiche per sviluppatori}
	\framesubtitle{Attiva/Disattiva funzionalità [Feature Toggles] (2)}

	% decrease font size for code listing
	\lstset{basicstyle=\tiny\ttfamily}

	\begin{itemize}

		\item Le funzionalità sono documentate per il core di TYPO3\newline
			\small
				(add link)
			\normalsize

		\item Gli sviluppatori di estensioni possono usare l'API per qualsiasi funzione	
			personalizzata fornita dall'estensione:

			\begin{lstlisting}
				if (GeneralUtility::makeInstance(Features::class)->isFeatureEnabled('myFeatureName')) {
				  // do custom processing
				}
			\end{lstlisting}

	\end{itemize}

\end{frame}

% ------------------------------------------------------------------------------
% LTXE-SLIDE-START
% LTXE-SLIDE-UID:		37566498-8a51e9a9-fd0e3914-106a267a
% LTXE-SLIDE-ORIGIN:	865e2894-3ce87c06-0f92cc8b-b880318e English
% LTXE-SLIDE-TITLE:		Additional Hook For Record List
% LTXE-SLIDE-REFERENCE:	61170-AddAdditionalHookForRecordList
% ------------------------------------------------------------------------------

\begin{frame}[fragile]
	\frametitle{Modifiche per sviluppatori}
	\framesubtitle{"Draw Header" Hook}

	% decrease font size for code listing
	\lstset{basicstyle=\tiny\ttfamily}

	\begin{itemize}
		\item Un nuovo hook è stato aggiunto a \texttt{EXT:recordlist}
			per renderizzare un contenuto sopra qualsiasi altro contenuto

		\item Per registare un hook:

			\begin{lstlisting}
				$GLOBALS['TYPO3_CONF_VARS']['SC_OPTIONS']['cms/layout/db_layout.php']
				  ['drawHeaderHook']['sys_note'] = \Vendor\Extkey\Hooks\PageHook::class . '->render';
			\end{lstlisting}

	\end{itemize}

\end{frame}

% ------------------------------------------------------------------------------
% LTXE-SLIDE-START
% LTXE-SLIDE-UID:		81c8038c-6990e031-025be7c9-ecab1cbd
% LTXE-SLIDE-ORIGIN:	40b574bb-9f828b9d-cecafb06-096f4f29 English
% LTXE-SLIDE-TITLE:		BE User Login Hook (1)
% LTXE-SLIDE-REFERENCE:	83529-ExecuteHooksOnBackendUserLogin
% ------------------------------------------------------------------------------

\begin{frame}[fragile]
	\frametitle{Modifiche per sviluppatori}
	\framesubtitle{BE User Login Hook (1)}

	% decrease font size for code listing
	\lstset{basicstyle=\tiny\ttfamily}

	\begin{itemize}
		\item Negli accessi di utenti di backend, sono eseguiti hook registrati
		\item Questo permette agli sviluppatori TYPO3 di creare funzioni che fanno
			\textit{qualcosa} quando accede un utente di BE

		\item I servizi di notifica sono tipici casi d'uso:

			\begin{itemize}
				\item Invia un messaggio a Slack o sistemi simili di messaggistica.
				\item Invia un SMS al cellulare dell'utente.
				\item Passa questo evento ad un altro sistema per monitorare attività sospette.
				\item ecc.
			\end{itemize}

	\end{itemize}

\end{frame}

% ------------------------------------------------------------------------------
% LTXE-SLIDE-START
% LTXE-SLIDE-UID:		d0e0656e-0011fe4a-4dd41e31-1858d7cc
% LTXE-SLIDE-ORIGIN:	8ab4904f-865ba977-d1d3cbd0-ff9b3797 English
% LTXE-SLIDE-TITLE:		BE User Login Hook (2)
% LTXE-SLIDE-REFERENCE:	83529-ExecuteHooksOnBackendUserLogin
% ------------------------------------------------------------------------------

\begin{frame}[fragile]
	\frametitle{Modifiche per sviluppatori}
	\framesubtitle{BE User Login Hook (2)}

	% decrease font size for code listing
	\lstset{basicstyle=\tiny\ttfamily}

	\begin{itemize}
		\item Per registrare un hook:

			\begin{lstlisting}
				$GLOBALS['TYPO3_CONF_VARS']['SC_OPTIONS']['t3lib/class.t3lib_userauthgroup.php']
				  ['backendUserLogin'][] = \Vendor\Extkey\Hooks\BackendUserLogin::class . '->dispatch';
			\end{lstlisting}

		\item Esegue il metodo \texttt{dispatch()} quando un utente di BE accede, e passa
			lo user array come parametro al metodo:

			\begin{lstlisting}
				public function dispatch($backendUser)
				{
				  if (isset($backendUser['user']['username'])) {
				    $username = $backendUser['user']['username'];
				    $email = $backendUser['user']['email'];
				    // do something...
				  }
				}
			\end{lstlisting}

	\end{itemize}

\end{frame}

% ------------------------------------------------------------------------------
