% ------------------------------------------------------------------------------
% TYPO3 Version 9.1 - What's New - Chapter "Deprecated Functions" (Italian Version)
%
% @author	Michael Schams <schams.net>
% @license	Creative Commons BY-NC-SA 3.0
% @link		http://typo3.org/download/release-notes/whats-new/
% @language	English
% ------------------------------------------------------------------------------
% LTXE-CHAPTER-UID:		3f842373-9262b8d3-f9c8de76-cf29ce17
% LTXE-CHAPTER-NAME:	Deprecated Functions
% ------------------------------------------------------------------------------

\section{Funzionalità deprecate/rimosse}
\begin{frame}[fragile]
	\frametitle{Funzionalità deprecate/rimosse}

	\begin{center}\huge{Capitolo 4:}\end{center}
	\begin{center}\huge{\color{typo3darkgrey}\textbf{Funzionalità deprecate/rimosse}}\end{center}

\end{frame}

% ------------------------------------------------------------------------------
% LTXE-SLIDE-START
% LTXE-SLIDE-UID:		e2b520e0-10d539f5-b6a85a38-c572095d
% LTXE-SLIDE-ORIGIN:	02a34a6e-add942aa-76ed434c-5651d314 English
% LTXE-SLIDE-TITLE:		Deprecated Usage of EXT:rsaauth
% LTXE-SLIDE-REFERENCE:	Deprecation-81852-DeprecatedUsageOfEXTrsaauth.rst
% ------------------------------------------------------------------------------

\begin{frame}[fragile]
	\frametitle{Funzionalità deprecate/rimosse}
	\framesubtitle{\texttt{EXT:rsaauth}}

	\begin{itemize}
		\item L'estensione \texttt{EXT:rsaauth} è stata segnata come \textbf{deprecata}
		\item Vista la rapida crescita dell'uso di SSL/TLS, la tecnologia utilizzata
			dall'estensione non è più considerata "sicura":

			\begin{itemize}
				\item Solo la password viene trasmessa crittografata
				\item Lo scambio di chiavi tra server e client non è autenticato\newline
					(consente attacchi man-in-the-middle)
				\item Gli ID di sessione sono trasmessi in chiaro, ma hanno lo stesso valore
					delle password
			\end{itemize}

		\item Va utilizzata una connessione sicura (HTTPS) al suo posto, e crittografati \underline{tutti i dati}
			scambiati tra client e server (TYPO3 frontend e backend)

	\end{itemize}

	\smaller
		\textbf{Nota: i browser moderni, di base, avvisano gli utenti quando i dati di un modulo sono inviati
			tramite una connessione non crittografata - non solo la password o i dati della carta di credito.}
	\normalsize

\end{frame}

% ------------------------------------------------------------------------------
% LTXE-SLIDE-START
% LTXE-SLIDE-UID:		19369fde-487bbce0-5200cedf-2af3fd0e
% LTXE-SLIDE-ORIGIN:	ca23c0b4-6e8960b4-aede1a9f-0c84574c English
% LTXE-SLIDE-TITLE:		EXT:Form YAML Configurations Typoscript Option
% LTXE-SLIDE-REFERENCE:	Deprecation-82089-ExtFormYamlConfigurationsTyposcriptOption.rst
% ------------------------------------------------------------------------------
%
%\begin{frame}[fragile]
%	\frametitle{Deprecated/Removed Functions}
%	\framesubtitle{YAML configuration of \texttt{EXT:form}}
%
%	% decrease font size for code listing
%	\lstset{basicstyle=\tiny\ttfamily}
%
%	\begin{itemize}
%		\item YAML configuration path has been marked as \textbf{deprecated}\newline
%			(will be removed in TYPO3 v10)
%
%			\begin{lstlisting}
%				plugin.tx_form {
%				  settings {
%				    yamlConfigurations {
%				      100 = EXT:my_site_package/Configuration/Yaml/CustomFormSetup.yaml
%				    }
%				  }
%				}
%			\end{lstlisting}
%
%		\item Instead, a single configuration file must be registered:
%
%			\begin{lstlisting}
%				plugin.tx_form {
%				  settings {
%				    configurationFile = EXT:my_site_package/Configuration/Yaml/CustomFormSetup.yaml
%				  }
%				}
%			\end{lstlisting}
%
%			\smaller
%				Same for backend setup (form editor): \texttt{module.tx\_form \{...\}}.
%			\normalsize
%
%	\end{itemize}
%
%\end{frame}

% ------------------------------------------------------------------------------
% LTXE-SLIDE-START
% LTXE-SLIDE-UID:		b0524a0f-d06fa534-fd920c96-f6187e6f
% LTXE-SLIDE-ORIGIN:	e82509d2-68646ba8-c887a63b-60ae97db English
% LTXE-SLIDE-TITLE:		Deprecate Unneeded RawValidator
% LTXE-SLIDE-REFERENCE:	83503-DeprecateUnneededRawValidator.rst
% ------------------------------------------------------------------------------

\begin{frame}[fragile]
	\frametitle{Funzionalità deprecate/rimosse}
	\framesubtitle{RawValidator}

	\begin{itemize}
		\item \texttt{RawValidator} è stato segnato come \textbf{deprecato}
		\item Era stato pensato per essere una sorta di NullObject per prevenire un "NoSuchValidatorException",
			ma queste eccezioni sono intercettate, il che rende obsoleto il validatore
		\item Visto che il validatore non convalida nulla, è molto probabile
			che questo cambiamento non influisca su alcuna installazione
		\item Nel caso uno sviluppatore utilizzi \texttt{RawValidator}, dovrà implementarlo autonomamente
	\end{itemize}


\end{frame}

% ------------------------------------------------------------------------------
